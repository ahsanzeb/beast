
\documentclass[a4paper,prb]{revtex4-1}  %,twocolumn
%\documentclass[preprint]{revtex4-1}

\usepackage{graphicx}  % needed for figures
\usepackage{dcolumn}   % needed for some tables
\usepackage{bm}        % for math
\usepackage{amssymb}   % for math
\usepackage{hyperref}

\usepackage{soul,color}

\usepackage{caption} %For two images side by side
\usepackage{subcaption} %For two images side by side

\usepackage{amsfonts}

\usepackage{amsmath}
%\usepackage{kbordermatrix}
\usepackage{blkarray}
\usepackage{braket}
\usepackage{multirow}
\usepackage{mathtools}

% for figure (Cartoon)
\usepackage{tikz}
\usepackage[utf8]{inputenc}

\newcolumntype{L}{>{$}l<{$}} % math-mode version of "l" column type


\newcommand{\y}{\lambda}
\newcommand{\yp}{\lambda^\prime}
\newcommand{\ph}{\Phi}
\newcommand{\ps}{\Psi}
\newcommand{\w}{E}
\newcommand{\cw}{\Omega}
\newcommand{\dl}{\delta}
\newcommand{\x}{\times}
\newcommand{\llra}{\Longleftrightarrow}
\newcommand{\ep}{\varepsilon}
\newcommand{\sig}{\sigma}
\newcommand{\al}{\alpha}
\newcommand{\be}{\beta}

\newcommand{\eps}{\epsilon}
\newcommand{\wrr}{\Omega_R}

\newcommand{\sx}{\mathcal{\hat S}_0}
\newcommand{\sy}{\mathcal{\hat S}_k}


\newcommand{\N}{\mathcal{N}}
\newcommand{\nb}{N}
\newcommand{\nd}{N_D}
\newcommand{\Nd}{\mathcal{N}_D}
\newcommand{\Nb}{\mathcal{N}_B}



\newcommand{\stt}{\hat{\mathcal{S}}}
\newcommand{\stf}{\hat{\tilde{\mathcal{S}}}}

\newcommand{\ws}{{\omega^\prime}}

\newcommand{\ev}{\hat{\Psi}^+(\w)}



\newcommand{\stket}[1]{{\mathcal{S}_{#1}}}

\newcommand{\nex}{\hat{\mathcal{N}}_{ex}}

\newcommand{\h}{\mathcal{\hat H}}
\newcommand{\z}{\mathcal{Z}_{\alpha}}
\newcommand{\hi}{\hat h_{\alpha}}
\newcommand{\hj}{\mathcal{\hat H}_j}

\newcommand{\hd}{\mathcal{\hat H}_{\Delta}}

\newcommand{\com}[1]{}

%\DeclareMathOperator{\Tr}{Tr}
\DeclareMathOperator{\diag}{diag} 
\DeclareMathOperator{\integer}{integer}
%\DeclareMathOperator{\nd}{and}
\DeclareMathOperator{\eV}{eV}
%\everymath{\displaystyle}

\newcommand{\ad}{\hat{a}^\dagger}
\newcommand{\an}{\hat{a}^{}}
\newcommand{\sd}{\hat{\sigma}^{+}_i}
\newcommand{\sn}{\hat{\sigma}^{-}_i}

\newcommand{\ua}{\uparrow}

\newcommand{\fd}{f(\hat{b}^\dagger)}
\newcommand{\fn}{f(\hat{b}^{})}


\newcommand{\mbf}[1]{\mathbf{#1}}
\newcommand{\del}{\hat{\Delta}^{}}



\newcommand{\fref}[1]{Fig.~(\ref{#1})}
\newcommand{\tref}[1]{Table~\ref{#1}}
\newcommand{\eref}[1]{Eq.~(\ref{#1})}

\newcommand{\rev}[1]{{\color{blue}{#1}}}  
\newcommand{\az}[1]{{\color{magenta}{#1}}} %Ahsan Zeb
\newcommand{\ha}[1]{{\color{red}{#1}}} 
\newcommand{\sm}[1]{{\color{brown}{#1}}} 

\newcommand{\Tau}{\mathrm{T}}

\DeclareMathOperator\erf{erf}
\DeclareMathOperator\erfi{erfi}
\DeclareMathOperator\ArcTanh{ArcTanh}
\DeclareMathOperator\sgn{sign}

\newcommand{\ylm}[1]{Y_{#1}(\hat r)} 


\def\Xint#1{\mathchoice
   {\XXint\displaystyle\textstyle{#1}}%
   {\XXint\textstyle\scriptstyle{#1}}%
   {\XXint\scriptstyle\scriptscriptstyle{#1}}%
   {\XXint\scriptscriptstyle\scriptscriptstyle{#1}}%
   \!\int}
\def\XXint#1#2#3{{\setbox0=\hbox{$#1{#2#3}{\int}$}
     \vcenter{\hbox{$#2#3$}}\kern-.5\wd0}}
\def\ddashint{\Xint=}
\def\dashint{\Xint -} % \sim


\newcommand\scalemath[2]{\scalebox{#1}{\mbox{\ensuremath{\displaystyle #2}}}}


\begin{document}

\title{Cooperative Jahn-Teller effect with rigid octahedral rotations in perovskites}

%\title{Orbital ordering in perovskites transition metal oxides with rigid octahedral rotations} 

\author{Rukhshanda Naheed}
\affiliation{Department of Physics, Quaid-i-Azam University, Islamabad 45320, Pakistan}

\author{M. Ahsan Zeb}
\affiliation{Department of Physics, Quaid-i-Azam University, Islamabad 45320, Pakistan}

\date{\today}
\begin{abstract}

Rigid octahedral rotations in a perovskite structure 
deform the crystal field ... from the symmetric octahedral ...


AB ions: cubic and octahedral cages....
O ions: octahedral cage....
Cubic structure: octahedral fields of AB and O are aligned... oriented 
Tilting and rotation of the O octahedra.... change their relative orientation....
and the the combined field is no more octahedral but has a lower symmetry. 
AB lattice: V: -10 to 17 percent of V of O lattice....
So... it is a strong perturbation to the octahedral field... and couples eg and t2g manifolds... and in lifts their degeneracies.... 




\end{abstract}
\maketitle




 
\section{Formalism: Crystal field Hamiltonian using multipole expansion of the crystal field potential}

If we have an arrangement of ions with charge $q$ around a given site,
we can express their (crystal field) potential in terms of spherical harmonics~(see \ref{append:multipoles}), 
\begin{align}
V(\vec r) &= \frac{q}{a}\sum_{lm}V_{l,m}~\left(\frac{r}{a}\right)^l \ylm{l,m},
\end{align}
and calculate the crystal field hamiltonian of the central ion/atom.
Assuming we only have d-orbitals ($l=2$) on the central atom,
the matrix elements of the crystal field Hamiltonian of this system 
are given by,
\begin{align}
H_{m',m''} &= -\frac{q}{a}\sum_{l,m} V_{l,m} C_{l}^{m'm''m} D_{l} ,
\end{align}
where 
$C_{l}^{m'm''m}=\braket{Y_{2,m'}| Y_{l,m}| Y_{2,m''}}$ are Gaunt coefficients for the given (real or complex) spherical harmonics,
and $D_{l}=\braket{(r/a)^l}$ is the expectation value for the d-orbitals, assumed to be the same for all d-orbitals, i.e., the radial part of all d-orbitals is assumed to the same~\cite{paxton-notes}.

\subsection{Atom/ion in an octahedral field: $t_{2g}$  and $e_g$ manifolds}
\label{sec:egt2g}
Consider six O ions each of charge $-q_o$ making an octahedron of size $a$ (placed at the face centres of a cube of edge length $a$).
The octahedral field at the centre of the octahedron or cube  
is given by~\cite{pavariniChap},
\begin{align}
\label{eq:voct}
V_{oct}(\vec r) = -\frac{224\sqrt{\pi}}{3}\frac{q_o}{a} \left(\frac{r}{a}\right)^4\left[\ylm{4,0}+\sqrt{\frac{5}{7}}\ylm{4,4} \right].
\end{align}
%where the monopole and $l>4$ components have been dropped
The crystal field Hamiltonian of a transition metal ion B at the centre 
turns out to be diagonal
and splits the energy levels into two manifolds,
$H_{m,m} = -\Delta/3$ for $m\in\{-2,-1,1\}$ ($xy,yz,zx$ orbitals)
called $t_{2g}$,
and 
$H_{m,m} = +2\Delta/3$ for $m\in\{2,0\}$ ($x^2-y^2,z^2$ orbitals)
called $e_g$.
The crystal field splitting between the two manifolds is
${\Delta=\frac{160}{3}\frac{q_o}{a} D_{4}}$.


\com{
$V_{oct}(\vec r)$ causes the d-orbitals to split into two manifolds,
$t_{2g}$  and $e_g$, where $t_{2g}$ contains three degenerate states ($xy,yz,zx$) while $e_g$ has two degenerate states ($x^2-y^2,z^2$).
The crystal field splitting between the two manifolds is
${\Delta=\frac{160}{3}\frac{q_o}{a} D_{4}}$.
\az{We will see how these degeneracies are lifted as we elongate (or compress)
the octahedron along a ....??}
}


\section{Janh-Teller Effect}

The Jahn--Teller effect~\cite{Jahn-Teller} describes the 
geometric and electronic instability of a %non-linear 
molecule, cluster or crystal with
a symmetric structure and orbitally degenerate ground state
%(i.e., except Kramers degeneracy)
against structural distortions that reduce the symmetry, lift the degeneracy, thereby
lowering the energy of the system.
 

\subsection{Tetragonal distortion of an octahedron}

Let's consider elongation of an octahedron in sec.~\ref{sec:egt2g} 
along one of its O-B-O bond. 
Align the cartesian axes along O-B-O bonds and stretch the two bonds along z-axis equally by an amount $a\delta$.
%where $a$ is the size of the octahedron (the distance between the O-O along any cartesian axes).
The electric potential at the central B-site (JT ion) of this deformed octahedron can be calculated 
and expanded in terms of (real) spherical harmonics as
\begin{align}
V^{JT}(\vec r) &= \frac{-q_o}{a}\sum_{lm}V_{l,m}^{JT}~\left(\frac{r}{a}\right)^l \ylm{l,m}.
\end{align}
%where the charge on O ion is taken as $-q_o$ ($q_o>0$, for brevity and clarity).
Only few multipole components are non-zero due to their strict symmetries.
The components %$V_{l,m}$ 
up to $l=4$, which are relevant for d-orbitals with $l=2$, are given by
~(see \ref{append:multipoles})
\begin{align}
V_{0,0}^{JT}=& \frac{8 \sqrt{\pi } (2 \delta +3)}{\delta +1},\\
V_{2,0}^{JT}=& -\frac{32 \sqrt{\frac{\pi }{5}} \delta  \left(\delta ^2+3 \delta +3\right)}{(\delta +1)^3},\\
V_{4,0}^{JT}=& \frac{32}{3} \sqrt{\pi } \left(\frac{4}{(\delta +1)^5}+3\right),\\
V_{4,4}^{JT}=& \frac{32 \sqrt{35 \pi }}{3}.
\end{align}
%The d-orbitals are not affected by odd-$l$ components either, here we do not have them anyway.
%(it's non-spherical components)
\rev{$V_{0,0}^{JT}$ rigidly shifts the whole spectrum 
[by $\simeq 4 \sqrt{\pi } (3-\delta )q_o/a$, 
favouring elongation ($\delta>0$) over
compression ($\delta<0$).... filling dependent argumetns... , might confuse??
]
and 
can be assumed to be contributing to the anharmonic part of the elastic energy of the system
that tends to prevent the structural distortion. 
We will ignore it and focus on the shifts and splittings that keep the average of the whole spectrum fixed.
}

\com{
At $\delta=0$, keeping only the non-spherical components, 
we obtain the
 octahedral potential for a symmetric octahedron~\cite{pavariniChap},
\begin{align}
\label{eq:voct}
\lim_{\delta\to0}V^{JT}(\vec r) \to V_{oct}(\vec r) = -\frac{224\sqrt{\pi}}{3}\frac{q_o}{a} \left(\frac{r}{a}\right)^4\left[\ylm{4,0}+\sqrt{\frac{5}{7}}\ylm{4,4} \right],
\end{align}
}



%\subsection{Crystal Field Splittings}


\az{The deformation $\delta$
will not couple these orbitals to each other either but 
modify the eg-t2g splitting as
${\Delta\to(1-\delta/3) \Delta }$,
and lift the degeneracies, as follows.
...
}
The crystal field Hamiltonian of the central B ion/JT ion
${H_{m',m''}^{JT} = (q_o/a)\sum_{l,m} V_{l,m}^{JT} C_{l}^{m'm''m} D_{l}}$
still turns out to be diagonal,
but the degeneracies of the two manifolds is lifted
and their average energies also shifted.
The $t_{2g}$ now splits into two levels,
with energies
\begin{align}
E_{xy} =&
\frac{32}{7} \left[
\left(1-\frac{1}{(\delta +1)^3}\right) D_2
-
\left(1-\frac{1}{(\delta +1)^5}\right) \frac{20}{9}D_4
\right]\frac{q_o}{a},\\
E_{yz} =& E_{zx} = -3/2 E_{xy},
\end{align}
as measured from their average 
%\begin{align}
$E_0^{t_{2g}} = -64q_oD_4 \left(2+(\delta +1)^{-5}\right)/9a$.
%\frac{64}{9} \left(-\frac{1}{(\delta +1)^5}-2\right) \text{D4}
%\end{align}
Similarly, $e_g$ splits up as
\begin{align}
E_{x^2-y^2}=&\frac{32}{7} \left[\left(1-\frac{1}{(\delta +1)^3}\right) \text{D2}+\frac{5}{3}\left(1-\frac{1}{(\delta +1)^5}\right) \text{D4}\right]\frac{q_o}{a},\\
E_{z^2}=&-E_{x^2-y^2},
\end{align}
measured from their average 
%\begin{align}
$E_0^{e_{g}} = 32 q_oD_4 \left(2+(\delta +1)^{-5}\right)/3a$.
%\frac{32}{3} \left(\frac{1}{(\delta +1)^5}+2\right) \text{D4}
%\end{align}
The crystal field splitting becomes
\begin{align}
\Delta \equiv E_0^{e_{g}}-E_0^{t_{2g}}=\frac{160}{9} \left(\frac{1}{(\delta +1)^5}+2\right) D_4.
\end{align}

At small $\delta$, 
we can expand the above expressions
 up to first order in $\delta$.
 $E_0^{t_{2g}}$, $E_0^{e_{g}}$ and $\Delta$ are rescaled by ${1-5\delta/3}$,
 and we can write
\begin{align}
(E_{xy},E_{yz},E_{zx})&= (2, -1, -1)
%\left[\frac{16}{63} (27 \text{D2}-100 \text{D4})\right]
\left[(D_2-\frac{100}{27}  D_4)\right]\Delta_{\delta},\\
(E_{x^2-y^2},E_{z^2})&=(2,-2)
\left[(D_2+\frac{25}{9} D_4)\right]\Delta_{\delta}
\end{align}
where 
${\Delta_{\delta}= \frac{48}{7}\frac{q_o}{a}\frac{\delta}{a}}$.


\com{
The crystal field splitting becomes
\begin{align}
E_0^{t_{2g}},E_0^{e_{g}},\Delta \to (1-\frac{5}{3}\delta)E_0^{t_{2g}}
E_0^{e_{g}} \to (1-\frac{5}{3}\delta) E_0^{e_{g}}
\Delta \to (1-\frac{5}{3}\delta)\Delta
\end{align}
}


\az{update the numbers,,, and quote realistic values of D2 and D4.}
Note the relative strengths of the two terms with $D_{2}$ and $D_{4}$.
\az{[refine argument]}
Typically, ${D_{2}\approx \sqrt{D_{4}}}$. 
Since
$D_{2}$ and $D_{4}$ are scaled with different 
powers of $a$ ($D_l=\braket{r^l}/a^l$),
their ratio depends on
the octahedron's size, i.e., $D_{2}/D_{4}\propto a^2$.
For well localised d-orbitals in a large octahedron,
${D_{2}/D_{4} > 25/9}$ so that elongation of the octahedron raises $xy$ and lowers $yz,xz$ states.
(Similarly, under such a condition, 
${D_{2}/D_{4} > 20/27}$ is automatically met so
it raises $x^2-y^2$ and lowers $z^2$).
However, if the d-orbitals are extended and octahedron is smaller,
a situation might result where ${20/27 < D_{2}/D_{4} < 25/9}$.
If so,
the ordering of the eg will be reversed as if a compression of the octahedron would have happened.
The extreme case of ${D_{2}/D_{4} < 20/27}$ (that seems less realistic)
would reverse both eg and t2g ordering for a given tetragonal distortion.
 

So far, we have calculated the changes in the single particle energy levels of the B ion. Whether the deformation would be energetically favourable or not will depend on the number of electrons occupying these levels. The magnitude of the deformation will be determined by a competition between the electronic energy gain and elastic energy required.

An example here a degen and a non degen case:.... 

\subsection{cooperative JT and orbital order}
just a brief description... 


\section{Model and calculations of ABO$_3$ Perovskites}%}

cubic ABO3 crystal Structure... 
rigid rotation/tilting: orthorhombic deformation... 
typically associated with the relative sizes of the A, B,O ions.~\cite{goodenoughPRL05}.


\subsection{Contribution of different ion cages in a cubic perovskite}
\label{sec:relativeVoctABO}

cube of A, octa of B, and octa of O.
%Assuming $q_o=-2$ and charge neutrality, $q_A+q_B + 3q_o=0$.

\com{
\begin{align}
%V_{oct}(\vec r) &= \frac{224\sqrt{\pi}}{3}\frac{q_o}{a} \left(\frac{r}{a}\right)^4\left[\ylm{4,0}+\sqrt{\frac{5}{7}}\ylm{4,4} \right],\\
V(\vec r) &= V_{O}(\vec r) + V_{A}(\vec r) + V_{B}(\vec r),\\
V_{O}(\vec r) &= V_{oct}(\vec r),\\
V_{A}(\vec r) &= \left(\frac{4}{81 \sqrt{3}}q_A \right)V_{oct}(\vec r),\\
V_{B}(\vec r) &= -\left(\frac{q_B}{64}\right)V_{oct}(\vec r).
\end{align}
It is interesting to note that 
$V_{A}(\vec r)$ and $V_{B}(\vec r)$ have opposite signs
and 
their combined effect
depends on their relative charges and vanishes
as
${q_A\to 1458/({256 \sqrt{3}+243})\simeq 2.124}$
where they cancel each other exactly.
}

\begin{align}
%V_{oct}(\vec r) &= \frac{224\sqrt{\pi}}{3}\frac{q_o}{a} \left(\frac{r}{a}\right)^4\left[\ylm{4,0}+\sqrt{\frac{5}{7}}\ylm{4,4} \right],\\
V(\vec r) &= V_{O}(\vec r) + V_{AB}(\vec r),\\
V_{O}(\vec r) &= V_{oct}(\vec r),\\
V_{AB}(\vec r) &= f_{q_A}V_{oct}(\vec r),\\
f_{q_A} &= \left(\frac{1}{32}+\frac{8}{81 \sqrt{3}}\right)\frac{q_A}{q_o} -\frac{3}{32},
\end{align}
where we have use the charge neutrality $q_B=3q_o-q_A$ in the last equation.
It is interesting to note that 
$V_{A}(\vec r)$ and $V_{B}(\vec r)$ have opposite signs
and 
their combined effect
depends on their relative charges and vanishes
as
${q_A\to 1458/({256 \sqrt{3}+243})\simeq 2.124}$
where they cancel each other exactly.


Including the contribution of these cages to the crystal field splitting $\Delta$, we obtain

\begin{align}
\Delta &= (1+f_{q_A})\Delta_O,
\end{align}
where
${\Delta_O=\frac{160}{3}\frac{q_o}{a} D_{4}}$.



\com{
\begin{align}
\Delta &= \left\{\frac{29}{32} + \left(\frac{8}{81 \sqrt{3}}+\frac{1}{32}\right)
 \left|\frac{q_A}{q_o}\right|  
\right\}\Delta_O,
\end{align}
}

We see that, relative to the O octahedron,
 the contribution of the
 two cation cages towards the total field splitting
 is relatively small at any $q_A$ but
 still large enough to be ignored.
It can be measured as $\Delta_{AB}/\Delta_O=(\Delta-\Delta_O)/\Delta_O$, 
which linearly changes from 
$-\frac{3}{32}\simeq -10\%$ at $q_A=0$
to $\frac{8}{27 \sqrt{3}}\simeq 17\%$ at $q_A=6$. 
Typical values of $q_A$ relevant for real transition metal oxides
can be safely assumed to range between $0$ and $4$;  
$q_A=0$ for systems like tungsten oxide $WO_3$ that do not have any cations at the A sites,
and $q_A=4$ still contributing just above $8\%$. 
% Delta_AB expression... and plot....
\az{Not just this pot, deformation induces $l=2$ V2m that can couple much more strongly.... bcs of larger D222. 
}


\subsection{Rigid octahedral rotations}

\subsection{Deformation of the crystal structure}



Octahedra rotation, and resulting unit cell changes:\\
bases rotation:\\
\az{Rotation matrices $R$ and $\mathcal{R}$ given in appendix.}

Coordinates of the Oxygen atoms transform according to 
the rotation matrix $R$.
While the d-orbitals basis states attached to the octahedron transform according to
$\mathcal{R}$.

Four adjesent octahedra in two layers along z have to rotate/tilt simultaneously in different directions... 

\az{figure vesta structure, label octahedra 1-4.}

We consider a
$\sqrt{2}\times\sqrt{2}\times 2$ unit cell of the cubic structure that transforms
to the unit cell of the orthorhombic structure under octahedral rotation and tilting.
Assuming $\vec a_1,\vec a_2,\vec a_3$ to be the lattice vectors,
two A and two B atoms in a layer are situated at
$(\vec a_1 + \vec a_3/2)/2$, $(\vec a_2+ \vec a_3/2)/2$,
and $(0,0,0)$, $(\vec a_1 + \vec a_2)/2$, 
while similar atoms in the second layer are shifted by $\vec a_3/2$ from these.
While the lattice vectors change under the rotation and tilting of the octahedra, 
these relations do not change.
%The B atom at the origin will be considered as the central JT atom.
The O atoms in the cubic structure lie half way between the B atoms and make octahedral cages around them.
Their positions in the deformed structure (see appendix) are not directly required for our calculations and discussion.


We rotate the octahedra by an angle $\beta$ about $\hat z \equiv 001$
and then tilt it by rotating it about new $110$ by an angle $\alpha$.
Four octahedra have to be rotated and tilted clockwise or counter closckwise
consistently. For example, %$\alpha  \{1,-1,-1,1\}$, $\beta  \{1,-1,1,-1\}$.
$(\alpha,\beta)$,
$(-\alpha,-\beta)$,
$(-\alpha,\beta)$,
$(\alpha,-\beta)$
for the atoms labelled $1$ through $4$ in Fig.~\ref{fig:structure}.
The structure transforms from cubic to orthorhombic under these rotation and tilting of the octahedra.
The lattice vectors of the deformed structure and the positions of A and B atoms can be calculated easily by calculating the coordinates of all six oxygen atoms that make the octahedra (by rotating and tilting the octahedra) for all four octahedra and 
comparing them for different octahedra for the shared atoms.
We obtain
% lattice vectors:
\begin{align}
\left(
\begin{array}{c}
\vec a_1 \\
\vec a_2\\
\vec a_3\\
\end{array}
\right)
= a
\left(
\begin{array}{ccc}
 \cos \beta  & \cos \beta  & 0 \\
 -\cos \alpha  \cos \beta  & \cos \alpha  \cos \beta  & 0 \\
 0 & 0 & 2 \cos \alpha  \\
\end{array}
\right)
\end{align}


\subsection{Deformation of the crystal field}

There are two different ways the crystal field $V(\vec r)$
 is deformed from the ideal case, $V_{voct}(\vec r)$.
\begin{enumerate}
\item The relative rotations of the AB and O frames due to
the rigid rotation and tilting of the oxygen octahedra removes the
choice of a set of d-orbitals where the effect of 
AB and O cages is the same. 
We no longer just get eg-t2g in any frame but with splittings induced by the rotation.
This can generate non-octahedral but still $l=4$ field components. 
\item
The rigid rotation and tilting of the octahedra keeps the octahedra symmetric but deforms the unit cell and AB ionic cages with it.
This not only changes the $l=4$ components but also induces 
$l=2$ field components
that can couple to the JT ion much more strongly (see discussion on $D_2$ vs $D_4$ below).
So, the magnitude of the net effect of the octahedral rotations is not limited by the relative sizes of the octahedral fields of AB and O in the cubic case (see sec.~\ref{sec:relativeVoctABO}).
\end{enumerate}


Since $V_{O}(\vec r)$ is relatively stronger, 
we can take the octahedra's local frames (that are rotated with them)
to describe the effect of $V_{O}(\vec r)$ first 
(leading to eg-t2g manifolds with a splitting of $\Delta_O$) 
and consider the effect of $V_{AB}(\vec r)$ on these later.
$V_{AB}(\vec r)$ will then couple and mix the orbitals in the two manifolds.
Since $\Delta_AB << \Delta_O$,
we can even ignore the coupling between the states belonging to the 
different manifolds.
This also makes it possible to analytically solve the problem in two limiting cases discussed later.



The total crystal field potential at the central B atom that we consider
 is given by
\begin{align}
V(\vec r) &= V_O(\vec r) + V_{AB}(\vec r),
\end{align}
where
$V_{AB}(\vec r)=V_{A}(\vec r)+V_{B}(\vec r)$
is due to A and B ion cages.
\az{repetition.... cubic case has similar expressions.... }


It is worth reminding that
$V(\vec r) $ a scalar quantity in the position space but a vector 
%$V$ is as a vector %of length $(l_{max}+1)^2=25$ for $l_{max}=4$.
in the spherical harmonics basis, so
we can calculate it in one basis as per our convenience and then transform it to another.
%Its components in the new basis.
\begin{align}
V(\vec r) &= \sum_{l,m} V_{lm} r^l Y_{lm}(\hat r)
= \sum_{l,m} \tilde V_{lm} r^l \tilde Y_{lm}(\hat r),\\
\tilde V_{lm} &= \sum_{l',m'}\mathcal{R}_{lm,l'm'}V_{l'm'},
\end{align}
where 
$\mathcal{R}(\alpha,\beta)$ is the rotation matrix for real spherical harmonics that we use. It can be calculated using the method described in Ref.~\cite{evanic}.

The multipole components of $V_O(\vec r)$ in the rotated frame does not change with the rigid octahedral rotations because, by definition, the octahedron itself rotates with its frame.
These can be read off Eq.~\ref{eq:voct}.
Similarly, its easier to obtain the components of $V(\vec r)^{AB}$
in the unrotated frame so we do that first and later transforms them to obtain the rotated frame components.
That is,
\begin{gather}
\tilde V_{lm} = \tilde V_{lm}^O + \sum_{l',m'}\mathcal{R}_{lm,l'm'} V_{l'm'}^{AB},\\
V_{l'm'}^{AB} = \left(\frac{q_A}{a} \right)V_{l'm'}^{A}   
+ \left(\frac{q_B}{a}\right)V_{l'm'}^{B}.
\end{gather}
%where $q_A + q_B = 3q_o$ ensures charge neutrality.


$\tilde V_{lm}^O$ are simply octahedral field components $l=4,m=0,4$ (see Eq.~\ref{eq:voct}).
However, $V_{lm}^{AB}$ contains the effect of the deformation of the crystal structure on two ion cages due to the octahedral rotations
so it now has not only other $l=4$ components but also $l=2$ components; see below.



\com{
\begin{table}[htp]
\caption{default}
\begin{center}
\begin{tabular}{|L|L|}
\hline
V^A_{2,-2}& 64 \sqrt{\frac{3 \pi }{5}} \cos ^2\beta  \left(\frac{1}{\left(\cos ^2\alpha +2 \cos ^2\beta \right)^{5/2}}-\frac{\sec ^3\alpha }{(\cos 2 \beta +2)^{5/2}}\right)\\ \hline
V^A_{2,0}& 32 \sqrt{\frac{\pi }{5}} \left(\frac{2 \sec ^3\alpha  \sin ^2\beta }{(\cos 2 \beta +2)^{5/2}}-\frac{32 \sqrt{\frac{\pi }{5}} (\cos 2 \beta -\cos 2 \alpha )}{\left(\cos ^2\alpha +2 \cos ^2\beta \right)^{5/2}}\right)\\ \hline
V^A_{4,-2}& \frac{128}{3} \sqrt{5 \pi } \cos ^2\beta  \left(\frac{3 \cos 2 \alpha -\cos 2 \beta +2}{\left(\cos ^2\alpha +2 \cos ^2\beta \right)^{9/2}}+\frac{\sec ^5\alpha  (\cos 2 \beta -5)}{(\cos 2 \beta +2)^{9/2}}\right)\\ \hline
V^A_{4,0}& -\frac{16}{3} \sqrt{\pi } \left(\frac{8 \cos 2 \alpha  (3 \cos 2 \beta +2)-2 \cos 4 \alpha+12 \cos 2 \beta -3 \cos 4 \beta +9}{\left(\cos ^2\alpha +2 \cos ^2\beta \right)^{9/2}}\right. \\ 
&+\left. \frac{\sec ^5\alpha  (36 \cos 2 \beta -3 \cos 4 \beta +23)}{(\cos 2 \beta +2)^{9/2}}\right) \\ \hline
V^A_{4,4}&\frac{1}{3} (-128) \sqrt{35 \pi } \cos ^4\beta  \left(\frac{1}{\left(\cos ^2\alpha +2 \cos ^2\beta \right)^{9/2}}+\frac{\sec ^5\alpha }{(\cos 2 \beta +2)^{9/2}}\right)\\ \hline

\end{tabular}
\end{center}
\label{default}
\end{table}%
}


\rev{$V(r)=q\sum_{i}(\sqrt{ ..... r\cos... })^{-1}$}

\az{full potential of the charges in deformed cage: manually refined}:
\begin{align}
V^A_{2,-2}&= 64 \sqrt{\frac{3 \pi }{5}} \cos ^2\beta  \left(\frac{1}{\left(\cos ^2\alpha +2 \cos ^2\beta \right)^{5/2}}-\frac{\sec ^3\alpha }{(\cos 2 \beta +2)^{5/2}}\right),\\
V^A_{2,0}&= 32 \sqrt{\frac{\pi }{5}} \left(\frac{2 \sec ^3\alpha  \sin ^2\beta }{(\cos 2 \beta +2)^{5/2}}-\frac{32 \sqrt{\frac{\pi }{5}} (\cos 2 \beta -\cos 2 \alpha )}{\left(\cos ^2\alpha +2 \cos ^2\beta \right)^{5/2}}\right),\\
V^A_{4,-2}&= \frac{128}{3} \sqrt{5 \pi } \cos ^2\beta  \left(\frac{3 \cos 2 \alpha -\cos 2 \beta +2}{\left(\cos ^2\alpha +2 \cos ^2\beta \right)^{9/2}}+\frac{\sec ^5\alpha  (\cos 2 \beta -5)}{(\cos 2 \beta +2)^{9/2}}\right),\\
V^A_{4,0}&= -\frac{16}{3} \sqrt{\pi } \left(\frac{8 \cos 2 \alpha  (3 \cos 2 \beta +2)-2 \cos 4 \alpha+12 \cos 2 \beta -3 \cos 4 \beta +9}{\left(\cos ^2\alpha +2 \cos ^2\beta \right)^{9/2}}\right. \\
&+\left. \frac{\sec ^5\alpha  (36 \cos 2 \beta -3 \cos 4 \beta +23)}{(\cos 2 \beta +2)^{9/2}}\right),\\
V^A_{4,4}&=\frac{1}{3} (-128) \sqrt{35 \pi } \cos ^4\beta  \left(\frac{1}{\left(\cos ^2\alpha +2 \cos ^2\beta \right)^{9/2}}+\frac{\sec ^5\alpha }{(\cos 2 \beta +2)^{9/2}}\right),
\end{align}


\com{
\begin{table}[htp]
\caption{default}
\begin{center}
\begin{tabular}{|L|}
\hline
V^{B}_{2,-2}=  \frac{64 \sqrt{\frac{3 \pi }{5}} \sin ^2\alpha  \sec ^3\beta }{(\cos 2 \alpha +3)^{5/2}}\\ %\hline
V^{B}_{2,0}=  4 \sqrt{\frac{\pi }{5}} \left(\sec ^3\alpha -\frac{8 \sec ^3\beta }{(\cos 2 \alpha +3)^{3/2}}\right)\\ %\hline
 V^{B}_{4,-2}=  -\frac{128 \sqrt{5 \pi } \sin ^2\alpha  \sec ^5\beta }{3 (\cos 2 \alpha +3)^{7/2}}\\ %\hline
V^{B}_{4,0}= \frac{4}{3} \sqrt{\pi } \left(\frac{24 \sec ^5\beta }{(\cos 2 \alpha +3)^{5/2}}+\sec ^5\alpha \right)\\ %\hline
 V^{B}_{4,4}=  -\frac{16 \sqrt{35 \pi } (-20 \cos 2 \alpha +\cos 4 \alpha -13) \sec ^5\beta }{3 (\cos 2 \alpha +3)^{9/2}}\\ \hline
\end{tabular}
\end{center}
\label{default}
\end{table}%
}

\az{full potential of the charges in deformed cage: }:

\begin{align}
V^{B}_{2,-2}&=  \frac{64 \sqrt{\frac{3 \pi }{5}} \sin ^2\alpha  \sec ^3\beta }{(\cos 2 \alpha +3)^{5/2}},
 \\ V^{B}_{2,0}&=  4 \sqrt{\frac{\pi }{5}} \left(\sec ^3\alpha -\frac{8 \sec ^3\beta }{(\cos 2 \alpha +3)^{3/2}}\right),\\
 V^{B}_{4,-2}&=  -\frac{128 \sqrt{5 \pi } \sin ^2\alpha  \sec ^5\beta }{3 (\cos 2 \alpha +3)^{7/2}},\\ 
V^{B}_{4,0}&= \frac{4}{3} \sqrt{\pi } \left(\frac{24 \sec ^5\beta }{(\cos 2 \alpha +3)^{5/2}}+\sec ^5\alpha \right),\\
 V^{B}_{4,4}&=  -\frac{16 \sqrt{35 \pi } (-20 \cos 2 \alpha +\cos 4 \alpha -13) \sec ^5\beta }{3 (\cos 2 \alpha +3)^{9/2}}
 \end{align}















\com{
$V_{l,m}^{A/B}$ are multipole components
of the potentials ${V_{dipoles}^{A/B}(\vec r;\alpha,\beta)}$ due to the deformation of A/B cages.
We need to include only $l=2,4$ components, $l=0$ uniformly shifts the whole spectrum (and can be taken to be part of the elastic energy of the lattice, as in JT case). [refine these arguments...]
}

\subsection{Crystal Field Hamiltonian}

The Hamiltonian of the central ion (JT ion) including 
the crystal field of the AB cages 
in the octahedron's local frame (rotated frame)
are given by,
\begin{align}
\tilde H_{m',m''} &= \sum_{l,m} \tilde V_{l,m} \tilde C_{l}^{m'm''m} D_{l} ,
%C_{l}^{m'm''m} &= \braket{\tilde Y_{2,m'}|\tilde Y_{l,m}|\tilde Y_{2,m''}},
\end{align}
where %$C_{l}^{m'm''m}$ 
$\tilde C_{l}^{m'm''m}=\braket{\tilde Y_{2,m'}|\tilde Y_{l,m}|\tilde Y_{2,m''}}= C_{l}^{m'm''m}$. 
% are again the Gaunt coefficients for the real spherical harmonics.

%\begin{align}
% bcs ham for electron with charge =-e=-1; sign change here
%\tilde H = \tilde H^O -  q_A \tilde H^A + (3q_o+q_A) \tilde H^B,
%\end{align}


\section{Results: Single particle spectrum and JT instability}
 
\subsection{t2g manifold} %states and energies}

\az{Focus on t2g:}\\
Let's consider three cases separately, only rotation, only tilting, and both rotation and tilting.

\subsubsection{Only rotation: $\alpha=0,\beta\neq 0$}
If we ignore the coupling between the eg and t2g manifolds (between $xy$ and $x^2-y^2$ at $\alpha=0$[see if it's the same at non zero alhpa?]),
their Hamiltonians turn out to be diagonal.



% from HA
\begin{align}
%\tilde H^A_{xy} =& -2 \Delta_{t2g},~
%\tilde H^A_{yz} = \tilde H^A_{zx} =\Delta_{t2g},\\
% \Delta_{t2g}(\alpha=0,\beta)
%\Delta_{t2g} =&\left( \Delta_{2}  D_2 + \Delta_4 D_4\right),\\
(E_{xy},&E_{yz},E_{zx})= (2, -1, -1)
\left[\frac{}{} \Delta_{2}  D_2 + \Delta_4 D_4\right],\\
\Delta_{l} =&  \left(q_A \Delta_{l}^A + q_B\Delta_{l}^B\right)/a,~(l=2,4),\\
\Delta_{2}^A =& 
%from A
\frac{64 \sin ^2\beta }{7 (\cos 2 \beta +2)^{5/2}},\\ % simplifies.... from:
 %\frac{16 (3 \sin \beta+\sin 3\beta)^2}{7 (\cos 2\beta+2)^{9/2}},\\
\Delta_4^A =&
 \frac{160 \sin ^2\beta}{63 (\cos 2\beta+2)^{9/2}} \times \\
&  (125 \cos 2\beta+42 \cos 4\beta+7 \cos 6\beta+82),\\
% \Delta_4^A =&
% \frac{160 \sin ^2\beta (125 \cos 2\beta+42 \cos 4\beta+7 \cos 6\beta+82)}{63 (\cos 2\beta+2)^{9/2}},\\
 %from B
\Delta_{2}^B =& \frac{2}{7} \left(1-\sec ^3\beta\right),\\
  \Delta_4^B =&\frac{5}{126} \left((7 \cos 4\beta-3) \sec ^5\beta-4\right),\\
 %\left( \Delta_{2}  D_2 + \Delta_4 D_4\right),\\
 \end{align}
along with an overall shift
\begin{align}
E_{0} =& \left(q_A E_{0}^A + q_BE_{0}^B\right)D_4/a,\\
E_{0}^A =&
 \frac{8 (92 \cos 2\beta+24 \cos 4\beta+20 \cos 6\beta+5 \cos 8\beta+51)}{9 (\cos 2\beta+2)^{9/2}},\\
E_{0}^B=&
 -\frac{1}{18} \left((5 \cos 4\beta+3) \sec ^5\beta+4\right),\\
 \end{align}




\begin{align}
\end{align}









Ignoring the coupling between t2g and eg states due to their large energetic difference, we still see that the hamiltonian seems to couple yz and zx if written in terms of multipoles compoments.
 it's only when we substitute Vlm in terms of alpha beta, 
 we see that i
 
  and since in a degenrate manifold any linear combination can be taken as a valid eigenstate set, we we can take yz and zx as degenerate eigenstates.

$D_2/D_4\simeq 14$ at 
$a=7$ bohr and $2eV$ crystal field gap ($\braket{r^4}=11.5809$, $D_4\simeq0.005$, $D_2=\sqrt{D_4}\simeq0.07$.)

\az{$D_4=<r^4>=12.005$ at $a=7$ gives normalised $D_4=<r^4>/a^4=1/200$ that gives $\Delta=15/4 (q_o/a)$.}

\begin{figure}[htbp]
\begin{center}
\includegraphics[width=0.5\linewidth]{/Users/maz/repos/perovskite/calc/analytical-structure-param/fig-data/energy-vs-beta-alpha-0-width-row}
\caption{$alpha=0$. 
}
\label{fig:evals-beta}
\end{center}
\end{figure}




\begin{figure}[htbp]
\begin{center}
\includegraphics[width=0.5\linewidth]{/Users/maz/repos/perovskite/calc/analytical-structure-param/fig-data/energy-vs-alpha-beta-0-width-row-character}
\caption{$beta=0$ case, $q_A=q_o/2$}
\label{fig:evals-alpha}
\end{center}
\end{figure}


\rev{

$\eta=\left(\frac{a\Delta_{O}}{q_o}\right)=\frac{4}{15}$
at $D_4=1/200$\\

$a=7$ bohr, and $\braket{r^4}=11.5809$ gives $2eV$ crystal field gap: $D_4=11.5809/a^4=0.004823365264473\simeq0.005=1/200=5\times10^{-3}$,
$D_2=\sqrt{D4}=0.0694503$,
$D_2/D_4\simeq 14$.
}

\az{$D_2/D_4$ ranges between $15$-$20$ for typical gaps of $2$-$1$ eV and $a\simeq 7$ bohr (and $q_o=2e$). 
It decreases with increasing gap as $1/\sqrt(gap)$, but is still $10$ at $gap=3.5$eV.}






\subsubsection{Only tilting: $\alpha\neq 0,\beta = 0$}



The Hamiltonian still has a simple structure, 
\begin{align} 
\tilde H = 
\left(
\begin{array}{ccc}
 E_{xy} & g & -g \\
 g & E_{yz} & \lambda  \\
 -g & \lambda  & E_{yz} \\
\end{array}
\right),
\end{align}
where the matrix elements have relatively long expressions and are can be found in the appendix. An approximation at small $\alpha,\beta$ is given later in this section. 



$\tilde H$ can be simplified by rotating the basis in $\{\ket{yz},\ket{zx}\}$ space,
as follows.
\begin{align}
\ket{B} &= \left(\ket{yz} - \ket{zx}\right)/\sqrt{2},\\
\ket{D} &=  (\ket{yz} + \ket{zx})/\sqrt{2}.
\end{align}
The state $\ket{D}$ at energy ${E_D= E_{yz}+\lambda}$ decouples 
from the rest and the state $\ket{B}$ at energy ${E_{yz}-\lambda}$ obtains the enhanced collective coupling $\sqrt{2} g$, 
\com{
\begin{align}
\tilde H_{Bright} = 
\left(
\begin{array}{cc}
E_{xy} & \sqrt{2} g \\
\sqrt{2} g & E_{yz}-\lambda
\end{array}
\right).
\end{align}
The Hamiltonian can be solved analytically.
}
and can be solved analytically to
obtain the 
the eigenstates
\begin{gather}
\ket{+} = \cos\gamma \ket{xy} + \sin\gamma \left(\ket{yz} - \ket{zx}\right)/\sqrt{2},\\
\ket{-} = \sin\gamma \ket{xy} - \cos\gamma \left(\ket{yz} - \ket{zx}\right)/\sqrt{2},\\
\tan2\gamma = \sqrt{8}g/(E_{xy}-E_{yz}+\lambda)
\end{gather}
at 
energies
\begin{gather}
E_{\pm} = \frac{E_{xy}+E_{yz}-\lambda}{2}  \pm \sqrt{(E_{xy}-E_{yz}+\lambda)^2 + 8g^2 }.
%\delta = E_{xy}-e_z+\lambda.
\end{gather}


Since we are only interested in small tilt and rotation,
an expansion of the matrix elements of the Hamiltonian about 
%${\alpha=0=\beta}$
${\alpha,\beta=0}$
%${(\alpha,\beta)=(0,0)}$
to the leading order turns out to be quite good an approximation.
The matrix elements in this case simplify a lot.
Defining ${Q = -\left( q_A D_2, q_A D_4, q_B D_2,q_B D_4  \right)}$,
 we can write them as
\begin{align}
E_{xy}=& Q.
\left(\frac{64 \alpha ^2}{63 \sqrt{3}},\frac{512 \left(15 \alpha ^2+7\right)}{1701 \sqrt{3}},-\frac{3\alpha ^2}{7} ,-\frac{5 \alpha ^2}{7}-\frac{2}{3}\right)\\
E_{yz}=& Q.
\left(-\frac{32 \alpha ^2}{63 \sqrt{3}},-\frac{512 \left(25 \alpha ^2-7\right)}{1701 \sqrt{3}},\frac{3 \alpha ^2}{14},\frac{85 \alpha ^2}{42}-\frac{2}{3}\right)\\
g=& Q.
\left(\frac{80}{63} \sqrt{\frac{2}{3}} \alpha ^3,0,-\frac{15 \alpha ^3}{14 \sqrt{2}},-\frac{5 \alpha ^3}{2 \sqrt{2}}\right)\\
\lambda=& Q.
\left(-\frac{64 \alpha ^2}{63 \sqrt{3}},\frac{2560 \alpha ^2}{1701 \sqrt{3}},\frac{3 \alpha ^2}{7},-\frac{40 \alpha ^2}{21}\right).
%E_0=& Q.\left(-\frac{16 \alpha ^2}{21 \sqrt{3}},\frac{256 \left(15 \alpha ^2-14\right)}{1701 \sqrt{3}},\frac{9 \alpha ^2}{28},\frac{2}{3}-\frac{45 \alpha ^2}{28}\right),
%Q =& \left( q_A D_2, q_A D_4, q_B D_2,q_B D_4     \right).
%E_0(0,0)=& Q.
%\frac{1}{2} \left(\frac{4 \text{D4} \text{qB}}{3}-\frac{1024 \text{D4} \text{qA}}{243 \sqrt{3}}\right)
\end{align}


$E_D$ and $E_{\pm}$ relative to their average ${E_0= \left(E_{xy} + 2 E_{yz}\right)/3}$
are shown in Fig.~\ref{fig:evals-alpha}.
$E_0$ itself shifts 
with the tilting $\alpha$, 
reflecting 
the change in eg-t2g gap $\Delta$.
This shift in $E_0$ is
%from its value 
%$\frac{D_4}{3} \left(\frac{512q}{81 \sqrt{3}}-2 (6-q)\right)$\\
%$\left(\frac{2}{729} \left(256 \sqrt{3}+243\right) q-4\right)D_4 $
%at $\alpha=0=\beta$, 
also included in Fig.~\ref{fig:evals-alpha}.


























\com{
\begin{figure}[htbp]
\begin{center}
\includegraphics[width=0.48\textwidth]{/Users/maz/repos/perovskite/calc/analytical-structure-param/code-rigid-rotation/figs/phi-0-q-0-r}
\includegraphics[width=0.48\textwidth]{/Users/maz/repos/perovskite/calc/analytical-structure-param/code-rigid-rotation/figs/phi-0-q-1-r}
\includegraphics[width=0.48\textwidth]{/Users/maz/repos/perovskite/calc/analytical-structure-param/code-rigid-rotation/figs/phi-0-q-2-r}
\includegraphics[width=0.48\textwidth]{/Users/maz/repos/perovskite/calc/analytical-structure-param/code-rigid-rotation/figs/phi-0-q-3-r}
\caption{d224=11.; norm should be eq to 0.00482334; d222=sqrt d224=0.0694503, d222/d224=14.3988 here. Dark state shuld b labelled, as well as pm states.
update to q=1,3 and analytic... d224norm=0.2?? or update theta0 figs to reduce d224 to 0.005?}
\label{default}
\end{center}
\end{figure}
}




\subsubsection{Rotation and tilting: $\alpha,\beta\neq 0$}


The t2g Hamiltonian in the general case cannot be solve analytically
 so there is no longer any advantage of ignoring the coupling between 
 the eg and t2g manifolds.
We can thus keep these couplings and numerically solve 
the full Hamiltonian instead.

 

\begin{figure}[htbp]
\begin{center}
\includegraphics[width=0.5\linewidth]{/Users/maz/repos/perovskite/calc/analytical-structure-param/fig-data/energy-vs-beta-alpha-10-width-row-beta-0-15}
\caption{$\alpha=10^\circ$ case: exact numerical results. solution of full 5x5}
\label{default}
\end{center}
\end{figure}




\begin{figure}[htbp]
\begin{center}
\includegraphics[width=0.5\linewidth]{/Users/maz/repos/perovskite/calc/analytical-structure-param/fig-data/wts-energy-vs-beta-alpha-10-width-row-beta-0-15}
\caption{$\alpha=10^\circ$ case: exact numerical results. solution of full 5x5}
\label{default}
\end{center}
\end{figure}



\com{
\begin{figure}[htbp]
\begin{center}
\includegraphics[width=0.48\textwidth]{/Users/maz/repos/perovskite/calc/analytical-structure-param/code-rigid-rotation/figs/q-1-4-alpha-14deg-states-wts-vs-beta}
\caption{eigenstates: weights of xy, yz, zx.}
\label{default}
\end{center}
\end{figure}
}


\subsection{eg manifold}



2x2 system so analytical sol always possible...
either theta or phi 0 givens no couplings but just shifts and eg orbitals stays as the eigenstates....

both theta phi couples the two states..... expressions relatively longer, but, can be approximated by simpler expressions obtained by an expansion about zero angles.... 

simpler expressions of eigenstates and eigenvalues.... in terms of D2 and D4 terms of A and B.


presentation and discussion of eg splitting...
combined plots for the three cases w.r.t  angles... as three  rows
and two q values on the opposite side of the crossover...
so 2x3 panel figure.


\begin{align}
(H_{x^2-y^2},&H_{z^2}) = (1,-1)
\left[\frac{}{} \Omega_{2}  D_2 + \Omega_4 D_4\right],\\
\Omega_{l} =&  \left(q_A \Omega_{l}^A + q_B\Omega_{l}^B\right)/a,~(l=2,4),\\
\Omega_{2}^A =&  -\frac{64 \left(\alpha ^2 \left(7 \beta ^2+3\right)-6 \beta ^2\right)}{189 \sqrt{3}},\\
\Omega_{4}^A =& \frac{640 \left(\alpha ^2 \left(\beta ^2+16\right)-32 \beta ^2\right)}{1701 \sqrt{3}},\\
\Omega_{2}^B =&  \frac{3}{14} \left(\alpha ^2 \left(3 \beta ^2+2\right)-4 \beta ^2\right),\\
\Omega_{4}^B =& \frac{1}{168} (-5) \left(\alpha ^2 \left(37 \beta ^2+46\right)-92 \beta ^2\right)
\end{align}


\begin{align}
E_0 &= \left(q_A E_0^A +q_B E_0^B\right)D_4/a,\\
E_0^A &=
\frac{256 \left(5 \left(4 \beta ^2+1\right) \alpha ^2+5 \beta ^2-3\right)}{243 \sqrt{3}}\\
E_0^B &=
(-\frac{1}{12} 5 \left(7 \beta ^2+4\right) \alpha ^2-\frac{5 \beta ^2}{3}+1).
\end{align}


% important, dont delete: H_{x^2-y^2,z^2} in terms of \Lambda_{l}^A/B
% where \Lambda_{l}^A/B are approximated upto second order in alpha beta.  
\com{
\begin{align}
H_{x^2-y^2,z^2}&=
\left[\frac{}{} \Lambda_{2}  D_2 + \Lambda_4 D_4\right],\\
\Lambda_{l} =&  \left(q_A \Lambda_{l}^A + q_B\Lambda_{l}^B\right)/a,~(l=2,4),\\
\Lambda_{2}^A =&
\frac{256 \alpha ^2 \beta }{189},\\
\Lambda_{4}^A =&
-\frac{1}{567} \left(5120 \alpha ^2 \beta \right),\\
\Lambda_{2}^B =&\frac{1}{7} (-4) \sqrt{3} \alpha ^2 \beta ,\\
\Lambda_{4}^B =&\frac{10}{7} \sqrt{3} \alpha ^2 \beta
\end{align}
}

\begin{align}
H_{x^2-y^2,z^2}&=
\frac{2\alpha ^2 \beta}{567} 
\left(128 q_A (3 D_2-20 D_4) - 81 \sqrt{3} q_B (2 D_2 -5 D_4)\right)
%&=\frac{2\alpha ^2 \beta}{567} \left(384 ( D_2-\frac{20}{3} D_4)q_A  - 162 \sqrt{3} (D_2 -\frac{5}{2} D_4)q_B \right)
\end{align}


\com{
\begin{align}
\left(
\begin{array}{cc}
 -\frac{64 \left(\alpha ^2 \left(7 \beta ^2+3\right)-6 \beta ^2\right)}{189 \sqrt{3}} & \frac{640 \left(\alpha ^2 \left(\beta ^2+16\right)-32 \beta ^2\right)}{1701 \sqrt{3}} \\
 \frac{3}{14} \left(\alpha ^2 \left(3 \beta ^2+2\right)-4 \beta ^2\right) & \frac{1}{168} (-5) \left(\alpha ^2 \left(37 \beta ^2+46\right)-92 \beta ^2\right) \\
\end{array}
\right)
\end{align}

\begin{align}
mean=
\left(
\begin{array}{cc}
 0 & \frac{256 \left(5 \left(4 \beta ^2+1\right) \alpha ^2+5 \beta ^2-3\right)}{243 \sqrt{3}} \\
 0 & -\frac{1}{12} 5 \left(7 \beta ^2+4\right) \alpha ^2-\frac{5 \beta ^2}{3}+1 \\
\end{array}
\right)
\end{align}
}






\begin{figure}[htbp]
\begin{center}
\includegraphics[width=0.5\linewidth]{/Users/maz/repos/perovskite/calc/analytical-structure-param/fig-data/energy-wts-eg}
\caption{$alpha=10^\circ$ case, $q_A=q_o/2,2q_o$}
\label{fig:e-wts-eg}
\end{center}
\end{figure}







\subsection{Effect of the full lattice}

In a crystal, obviously not only the nearest AB ions matter but other ions at longer distances also participate in the effect described in the prev sec.

We can numerically calculate the multipole components of the potential of the full lattice by following the recipe in Ref.~\cite{PaxtonNotes, book?}
and Ewald summation method~\cite{Ewald}.
To do this, 
we borrow relevant routines from the Tight Binding code TBE [confirm the names] in Questaal~\cite{https://www.questaal.org/} that implements these methods.


First,
FL does not preserve the cubic symmetry of the potential and the degeneracy of the t2g manifold 
is slightly lifted at zero rotations, i.e., for a cubic structure.
However, it is insignificant and we will just ignore it.

What's more important is the .... 

Potential of full lattice does not change the qualitative picture.
just shifts  .... so considering only AB cages overestimate or underestimate... 

If one renormalises the charges qA and qB, a
quantitative agreement with the full lattice case can be obtained.








\begin{figure}[htbp]
\begin{center}
\includegraphics[width=0.5\linewidth]{/Users/maz/repos/perovskite/calc/analytical-structure-param/full-lattice-potential-tests/et2g-vs-q-beta-20}
\caption{$alpha=0$ evals vs q at $beta=20$. 
}
\label{fig:evals-q-beta}
\end{center}
\end{figure}










\section{Appendix: Rotation matrices}



\begin{gather}
\nonumber
R=
\left(
\scalemath{0.9}{
\begin{array}{ccc}
 \frac{1}{2} ((\cos \alpha+1) \cos \beta+(\cos \alpha-1) \sin \beta) & \frac{1}{2} (\cos \beta-\sin \beta-\cos \alpha (\cos \beta+\sin \beta)) & \frac{\sin \alpha (\cos \beta+\sin \beta)}{\sqrt{2}} \\
 \frac{1}{2} (-\cos \alpha \cos \beta+\cos \beta+\cos \alpha \sin \beta+\sin \beta) & \frac{1}{2} (\cos \beta+\cos \alpha (\cos \beta-\sin \beta)+\sin \beta) & \frac{\sin \alpha (\sin \beta-\cos \beta)}{\sqrt{2}} \\
 -\frac{\sin \alpha}{\sqrt{2}} & \frac{\sin \alpha}{\sqrt{2}} & \cos \alpha \\
\end{array}
}
\right),\\ \nonumber
\com{
\mathcal{R}=
\left(
\scalemath{0.7}{
\begin{array}{ccccc}
 R_{12} R_{21}+R_{11} R_{22} & R_{13} R_{22}+R_{12} R_{23} & \sqrt{3} R_{13} R_{23} & R_{13} R_{21}+R_{11} R_{23} & R_{11} R_{21}-R_{12} R_{22} \\
 R_{22} R_{31}+R_{21} R_{32} & R_{23} R_{32}+R_{22} R_{33} & \sqrt{3} R_{23} R_{33} & R_{23} R_{31}+R_{21} R_{33} & R_{21} R_{31}-R_{22} R_{32} \\
 \frac{-1}{\sqrt{3}}\left(R_{11} R_{12}+R_{21} R_{22}-2 R_{31} R_{32}\right) & \frac{-1}{\sqrt{3}}\left(R_{12} R_{13}+R_{22} R_{23}-2 R_{32} R_{33}\right)  & -\frac{1}{2} R_{13}^2+R_{33}^2-\frac{R_{23}^2}{2} & \frac{-1}{\sqrt{3}}\left(R_{11} R_{13}+R_{21} R_{23}-2 R_{31} R_{33}\right) & 0\\ % identically 0 in our case of specific rotation and tilt.
 %\frac{1}{2\sqrt{3}}\left(-R_{11}^2+R_{12}^2-R_{21}^2+R_{22}^2+2 R_{31}^2-2 R_{32}^2\right) \\
 R_{12} R_{31}+R_{11} R_{32} & R_{13} R_{32}+R_{12} R_{33} & \sqrt{3} R_{13} R_{33} & R_{13} R_{31}+R_{11} R_{33} & R_{11} R_{31}-R_{12} R_{32} \\
 R_{11} R_{12}-R_{21} R_{22} & R_{12} R_{13}-R_{22} R_{23} & \frac{1}{2} \sqrt{3} \left(R_{13}^2-R_{23}^2\right) & R_{11} R_{13}-R_{21} R_{23} & \frac{1}{2} \left(R_{11}^2-R_{12}^2-R_{21}^2+R_{22}^2\right) \\
\end{array}
}
\right),\\
}
\mathcal{R}=
\left(
\scalemath{0.65}{
\begin{array}{ccccc}
 \frac{1}{4} (\cos 2 \alpha+3) \cos 2 \beta & \frac{\sin \alpha (\cos \alpha \cos 2 \beta+\sin 2 \beta)}{\sqrt{2}} & -\frac{1}{2} \sqrt{3} \cos 2 \beta \sin ^2\alpha & \frac{\sin \alpha (\sin 2 \beta-\cos \alpha \cos 2 \beta)}{\sqrt{2}} & \cos \alpha \sin 2 \beta \\
 \frac{\cos \alpha \sin \alpha (\sin \beta-\cos \beta)}{\sqrt{2}} & \frac{1}{2} (\cos 2 \alpha (\cos \beta-\sin \beta)+\cos \alpha (\cos \beta+\sin \beta)) & \sqrt{\frac{3}{2}} \cos \alpha \sin \alpha (\sin \beta-\cos \beta) & \frac{1}{2} (\cos 2 \alpha (\sin \beta-\cos \beta)+\cos \alpha (\cos \beta+\sin \beta)) & -\frac{\sin \alpha (\cos \beta+\sin \beta)}{\sqrt{2}} \\
 -\frac{1}{2} \sqrt{3} \sin ^2\alpha & \sqrt{\frac{3}{2}} \cos \alpha \sin \alpha & \frac{1}{4} (3 \cos 2 \alpha+1) & -\sqrt{\frac{3}{2}} \cos \alpha \sin \alpha & 0 \\
 \frac{\cos \alpha \sin \alpha (\cos \beta+\sin \beta)}{\sqrt{2}} & \frac{1}{2} (\cos \alpha (\cos \beta-\sin \beta)-\cos 2 \alpha (\cos \beta+\sin \beta)) & \sqrt{\frac{3}{2}} \cos \alpha \sin \alpha (\cos \beta+\sin \beta) & \frac{1}{2} (\cos \alpha (\cos \beta-\sin \beta)+\cos 2 \alpha (\cos \beta+\sin \beta)) & \frac{\sin \alpha (\sin \beta-\cos \beta)}{\sqrt{2}} \\
 -\left(\cos ^2\alpha+1\right) \cos \beta \sin \beta & \frac{\sin \alpha (\cos 2 \beta-2 \cos \alpha \cos \beta \sin \beta)}{\sqrt{2}} & \sqrt{3} \cos \beta \sin ^2\alpha \sin \beta & \frac{\sin \alpha (\cos 2 \beta+\cos \alpha \sin 2 \beta)}{\sqrt{2}} & \cos \alpha \cos 2 \beta \\
\end{array}
}
\right)
\end{gather}



\com{

\begin{align}
\mathcal{R}=&
\left(
\begin{array}{ccccc}
 R_{-1,1} R_{1,-1}+R_{-1,-1} R_{1,1} & R_{-1,0} R_{1,-1}+R_{-1,-1} R_{1,0} & \sqrt{3} R_{-1,0} R_{1,0} & R_{-1,1} R_{1,0}+R_{-1,0} R_{1,1} & R_{-1,1} R_{1,1}-R_{-1,-1} R_{1,-1} \\
 R_{-1,1} R_{0,-1}+R_{-1,-1} R_{0,1} & R_{-1,0} R_{0,-1}+R_{-1,-1} R_{0,0} & \sqrt{3} R_{-1,0} R_{0,0} & R_{-1,1} R_{0,0}+R_{-1,0} R_{0,1} & R_{-1,1} R_{0,1}-R_{-1,-1} R_{0,-1} \\
 -\frac{R_{-1,-1} R_{-1,1}-2 R_{0,-1} R_{0,1}+R_{1,-1} R_{1,1}}{\sqrt{3}} & -\frac{R_{-1,-1} R_{-1,0}-2 R_{0,-1} R_{0,0}+R_{1,-1} R_{1,0}}{\sqrt{3}} & -\frac{1}{2} R_{-1,0}^2+R_{0,0}^2-\frac{R_{1,0}^2}{2} & -\frac{R_{-1,0} R_{-1,1}-2 R_{0,0} R_{0,1}+R_{1,0} R_{1,1}}{\sqrt{3}} & \frac{R_{-1,-1}^2-R_{-1,1}^2-2 R_{0,-1}^2+2 R_{0,1}^2+R_{1,-1}^2-R_{1,1}^2}{2 \sqrt{3}} \\
 R_{0,1} R_{1,-1}+R_{0,-1} R_{1,1} & R_{0,0} R_{1,-1}+R_{0,-1} R_{1,0} & \sqrt{3} R_{0,0} R_{1,0} & R_{0,1} R_{1,0}+R_{0,0} R_{1,1} & R_{0,1} R_{1,1}-R_{0,-1} R_{1,-1} \\
 R_{1,-1} R_{1,1}-R_{-1,-1} R_{-1,1} & R_{1,-1} R_{1,0}-R_{-1,-1} R_{-1,0} & \frac{1}{2} \sqrt{3} \left(R_{1,0}^2-R_{-1,0}^2\right) & R_{1,0} R_{1,1}-R_{-1,0} R_{-1,1} & \frac{1}{2} \left(R_{-1,-1}^2-R_{-1,1}^2-R_{1,-1}^2+R_{1,1}^2\right) \\
\end{array}
\right)
\end{align}


\begin{gather}
\mathcal{R}=
\left(
\scalemath{0.6}{
\begin{array}{ccccc}
 \frac{1}{4} (\cos 2 \alpha+3) \cos 2 \beta & \frac{\sin \alpha (\cos \alpha \cos 2 \beta+\sin 2 \beta)}{\sqrt{2}} & -\frac{1}{2} \sqrt{3} \cos 2 \beta \sin ^2\alpha & \frac{\sin \alpha (\sin 2 \beta-\cos \alpha \cos 2 \beta)}{\sqrt{2}} & \cos \alpha \sin 2 \beta \\
 \frac{\cos \alpha \sin \alpha (\sin \beta-\cos \beta)}{\sqrt{2}} & \frac{1}{2} (\cos 2 \alpha (\cos \beta-\sin \beta)+\cos \alpha (\cos \beta+\sin \beta)) & \sqrt{\frac{3}{2}} \cos \alpha \sin \alpha (\sin \beta-\cos \beta) & \frac{1}{2} (\cos 2 \alpha (\sin \beta-\cos \beta)+\cos \alpha (\cos \beta+\sin \beta)) & -\frac{\sin \alpha (\cos \beta+\sin \beta)}{\sqrt{2}} \\
 -\frac{1}{2} \sqrt{3} \sin ^2\alpha & \sqrt{\frac{3}{2}} \cos \alpha \sin \alpha & \frac{1}{4} (3 \cos 2 \alpha+1) & -\sqrt{\frac{3}{2}} \cos \alpha \sin \alpha & 0 \\
 \frac{\cos \alpha \sin \alpha (\cos \beta+\sin \beta)}{\sqrt{2}} & \frac{1}{2} (\cos \alpha (\cos \beta-\sin \beta)-\cos 2 \alpha (\cos \beta+\sin \beta)) & \sqrt{\frac{3}{2}} \cos \alpha \sin \alpha (\cos \beta+\sin \beta) & \frac{1}{2} (\cos \alpha (\cos \beta-\sin \beta)+\cos 2 \alpha (\cos \beta+\sin \beta)) & \frac{\sin \alpha (\sin \beta-\cos \beta)}{\sqrt{2}} \\
 -\left(\cos ^2\alpha+1\right) \cos \beta \sin \beta & \frac{\sin \alpha (\cos 2 \beta-2 \cos \alpha \cos \beta \sin \beta)}{\sqrt{2}} & \sqrt{3} \cos \beta \sin ^2\alpha \sin \beta & \frac{\sin \alpha (\cos 2 \beta+\cos \alpha \sin 2 \beta)}{\sqrt{2}} & \cos \alpha \cos 2 \beta \\
\end{array}
}
\right)
\end{gather}


}




\section{Appendix}
\label{append:multipoles}
\sm{
It is worth noting that $V_{l,m}^{A/B}$ can be found without 
evaluating the 
inner products between the spherical harmonics and
$V^{A/B}(\vec r;\alpha,\beta)$.
The latter can be computed 
using the simple textbook formula 
$V^{A/B}(\vec r;\alpha,\beta)=\sum_{i}1/r_i$, $\vec r_i$ is the position of the observation point $\vec r$ (measured from the JT ion/central B ion).
Once we have ${V^{A/B}(\vec r;\alpha,\beta)}$,
$V_{l,m}^{A/B}$ can be calculated by first expanding this potential
in powers of $r$, 
and then expanding the coefficient of $r^l$ 
in powers of $\cos\theta, \cos\phi$ (or $\sin\theta, \sin\phi$; or simply the spherical coordinates $\theta,\phi$ of $\hat r$) 
and comparing the terms with similar 
expansion of the formal expression on the right side of Eq.~\ref{eq:ylm_expansion} to get linear equations in $\{V_{l,m}\}$, and finally solving this system of equations.
The only relevant non-zero components of $V_{l,m}^{A/B}$ are the following.







It is worth noting that $V_{l,m}^{A/B}$ can be found without 
evaluating the 
inner products between the spherical harmonics and
$V^{A/B}(\vec r;\alpha,\beta)$.
The latter can be computed 
using the simple textbook formula 
$V^{A/B}(\vec r;\alpha,\beta)=\sum_{i}1/r_i$, $\vec r_i$ is the position of the observation point $\vec r$ (measured from the JT ion/central B ion).
Once we have ${V^{A/B}(\vec r;\alpha,\beta)}$,
$V_{l,m}^{A/B}$ can be calculated by first expanding this potential
in powers of $r$, 
and then expanding the coefficient of $r^l$ 
in powers of $\cos\theta, \cos\phi$ (or $\sin\theta, \sin\phi$; or simply the spherical coordinates $\theta,\phi$ of $\hat r$) 
and comparing the terms with similar 
expansion of the formal expression on the right side of Eq.~\ref{eq:ylm_expansion}.
The only relevant non-zero components of $V_{l,m}^{A/B}$ are the following.





\subsection{$beta=0$ case: analytical solution:}

$E_{xy},E_{yz},g,\lambda$ all have the same form as 
\begin{align}
X = -\frac{1}{a}\sum_{l=2,4}\left(q_A X^{A,l} + q_B X^{B,l}\right) D_l,
\end{align}
%where $X=E_{xy},E_{yz},g,\lambda$,
where various terms are given in the following.

%\begin{align}
%E_{xy} = -\frac{1}{a}\sum_{l=2,4}\left(q_A E_{xy}^{A,l} + q_B E_{xy}^{B,l}\right) D_l,
%\end{align}
%where $E_{xy}^{A/B,l}$ are given in the following.


%A D2
\begin{align}
E_{xy}^{A,2} =& \frac{4 \left(-72 \cos 2 \alpha -27 \cos 4 \alpha +\sqrt{6} \sqrt{\cos 2 \alpha +5} \left(15 \cos \alpha +\cos 3 \alpha -16 \sec ^3\alpha \right)+99\right)}{63 \left(\cos ^2\alpha +2\right)^{5/2}},\\
E_{yz}^{A,2} =&\frac{144 \cos 2 \alpha +54 \cos 4 \alpha +\sqrt{\frac{3}{2}} (9 \sin \alpha +\sin 3 \alpha )^2 \sqrt{\cos 2 \alpha +5} \sec ^3\alpha -198}{63 \left(\cos ^2\alpha +2\right)^{5/2}},\\
g^{A,2} =&\frac{2 \left(2 \sqrt{3} (\cos 2 \alpha +5)^{5/2}-216 \sqrt{2} \cos ^5\alpha \right) \tan \alpha  \sec \alpha }{63 \left(\cos ^2\alpha +2\right)^{5/2}},\\
\lambda^{A,2} =&\frac{216 \cos 4 \alpha -4 \sqrt{6} (\cos 2 \alpha +3) (\cos 2 \alpha +5)^{5/2} \sec ^3\alpha +3240}{252 \left(\cos ^2\alpha +2\right)^{5/2}},
\end{align}



% A D4
\begin{align}
E_{xy}^{A,4} =& \frac{1}{20412 \left(\cos ^2\alpha +2\right)^{9/2}}
\left[\frac{}{} -330480 \cos 2 \alpha +243 (2964 \cos 4 \alpha +80 \cos 6 \alpha +35 \cos 8 \alpha +5449)\right.\\
&\hspace{4cm} + \left. \sqrt{\frac{3}{2}} (180 \cos 2 \alpha -245 \cos 4 \alpha +513) (\cos 2 \alpha +5)^{9/2} \sec ^5\alpha \right],\\
E_{yz}^{A,4} =& \frac{}{10206 \left(\cos ^2\alpha +2\right)^{9/2}}
\left[660960 \cos 2 \alpha +243 (652 \cos 4 \alpha -160 \cos 6 \alpha -35 \cos 8 \alpha +407)\right.\\
&\hspace{4cm} + \left.\sqrt{\frac{3}{2}} (60 \cos 2 \alpha +245 \cos 4 \alpha -81) (\cos 2 \alpha +5)^{9/2} \sec ^5\alpha \right],\\
g^{A,4} =& \frac{5 \left(31104 \sqrt{2} \cos ^7\alpha  (8 \cos 2 \alpha +7 \cos 4 \alpha +145)-32 \sqrt{3} (\cos 2 \alpha +5)^{9/2} (49 \cos 2 \alpha -9)\right) \tan \alpha  \sec ^3\alpha }{163296 \left(\cos ^2\alpha +2\right)^{9/2}},\\
\lambda^{A,4} =& \frac{1}{10206 \left(\cos ^2\alpha +2\right)^{9/2}}
\left[5 \left(180792 \cos 2 \alpha +243 (212 \cos 4 \alpha +24 \cos 6 \alpha +7 \cos 8 \alpha +37)\right)\right.\\
&\hspace{4cm} + \left.\sqrt{\frac{3}{2}} (12 \cos 2 \alpha -49 \cos 4 \alpha -27) (\cos 2 \alpha +5)^{9/2} \sec ^5\alpha \right],
\end{align}



% B D2
\begin{align}
E_{xy}^{B,2} =& \frac{32 \cos 2 \alpha +24 \cos 4 \alpha -(3 \cos 2 \alpha +1) (\cos 2 \alpha +3)^{5/2} \sec ^3\alpha +72}{7 (\cos 2 \alpha +3)^{5/2}},\\
E_{yz}^{B,2} =&\frac{-32 \cos 2 \alpha -24 (\cos 4 \alpha +3)+(3 \cos 2 \alpha +1) (\cos 2 \alpha +3)^{5/2} \sec ^3\alpha }{14 (\cos 2 \alpha +3)^{5/2}},\\
g^{B,2} =&\frac{3 \left(40 \cos \alpha +20 \cos 3 \alpha -\sqrt{\cos 2 \alpha +3} (12 \cos 2 \alpha +\cos 4 \alpha )+4 \cos 5 \alpha -19 \sqrt{\cos 2 \alpha +3}\right) \tan \alpha  \sec \alpha }{7 \sqrt{2} (\cos 2 \alpha +3)^{5/2}},\\
\lambda^{B,2} =&-\frac{3 \left(8 \cos 2 \alpha +(\cos 2 \alpha +3)^{3/2} \tan ^2\alpha  \sec \alpha -8\right)}{7 (\cos 2 \alpha +3)^{3/2}},
\end{align}



%B D4
\begin{align}
E_{xy}^{B,4} &= \scalemath{0.6}{\frac{1}{42} \left(-\frac{4 (851 \cos 2 \alpha +10 \cos 4 \alpha -35 \cos 6 \alpha +198)}{(\cos 2 \alpha +3)^{7/2}}+19 \sec ^5\alpha -50 \sec ^3\alpha +35 \sec \alpha \right)},\\
E_{yz}^{B,4} &= \scalemath{0.6}{\frac{-3648 \cos 2 \alpha -8 (628 \cos 4 \alpha +120 \cos 6 \alpha +35 \cos 8 \alpha +297)-\frac{1}{2} (20 \cos 2 \alpha +35 \cos 4 \alpha +9) (\cos 2 \alpha +3)^{9/2} \sec ^5\alpha }{84 (\cos 2 \alpha +3)^{9/2}}},\\
g^{B,4} &= \scalemath{0.6}{\frac{5 (5 \sin \alpha +\sin 3 \alpha ) \left(-3024 \cos \alpha -1344 \cos 3 \alpha +216 \cos 7 \alpha +56 \cos (9 \alpha )+(1224 \cos 2 \alpha +676 \cos 4 \alpha +120 \cos 6 \alpha +7 \cos 8 \alpha ) \sqrt{\cos 2 \alpha +3}+21 \sqrt{\cos 2 \alpha +3}\right) \sec ^4\alpha }{672 \sqrt{2} (\cos 2 \alpha +3)^{9/2}}},\\
\lambda^{B,4} &= \scalemath{0.6}{\frac{5}{42} \left(-\frac{32 \sin ^2\alpha  (4 \cos 2 \alpha +7 \cos 4 \alpha +5)}{(\cos 2 \alpha +3)^{7/2}}-(7 \cos 2 \alpha +5) \tan ^2\alpha  \sec ^3\alpha \right).}
\end{align}








\com{
\begin{figure}[htbp]
\begin{center}
\includegraphics[width=0.5\linewidth]{/Users/maz/repos/perovskite/calc/analytical-structure-param/fig-data/energy-vs-alpha-beta-0-width-row}
\caption{$beta=0$ case, $q_A=q_o/2$}
\label{fig:evals-alpha}
\end{center}
\end{figure}
}





\begin{figure}[htbp]
\begin{center}
\includegraphics[width=0.48\textwidth]{/Users/maz/repos/perovskite/calc/analytical-structure-param/code-rigid-rotation/figs/theta-0-q-1-D24terms}
\includegraphics[width=0.48\textwidth]{/Users/maz/repos/perovskite/calc/analytical-structure-param/code-rigid-rotation/figs/theta-0-q-2-D24terms}
\includegraphics[width=0.48\textwidth]{/Users/maz/repos/perovskite/calc/analytical-structure-param/code-rigid-rotation/figs/theta-0-q-3-D24terms}
\caption{append fig; combine in one plot: $D_2/D_4\simeq 14$ at 
$a=7$ bohr and $2eV$ crystal field gap ($\braket{r^4}=11.5809$, $D_4\simeq0.005$, $D_2=\sqrt{D_4}\simeq0.07$.)
}
\label{default}
\end{center}
\end{figure}





\com{
\begin{figure}[htbp]
\begin{center}
\includegraphics[width=0.8\linewidth]{/Users/maz/repos/perovskite/calc/analytical-structure-param/fig-data/energy-vs-beta-alpha-0-width}
\caption{}
\label{default}
\end{center}
\end{figure}
}



\com{
\begin{figure}[htbp]
\begin{center}
\includegraphics[width=0.48\textwidth]{/Users/maz/repos/perovskite/calc/analytical-structure-param/fig-data/energy-vs-alpha-beta-0}
\caption{append fig; combine in one plot: $D_2/D_4\simeq 14$ at 
$a=7$ bohr and $2eV$ crystal field gap ($\braket{r^4}=11.5809$, $D_4\simeq0.005$, $D_2=\sqrt{D_4}\simeq0.07$.)
}
\label{default}
\end{center}
\end{figure}
}




\com{
\begin{figure}[htbp]
\begin{center}
\includegraphics[width=0.48\textwidth]{/Users/maz/repos/perovskite/calc/analytical-structure-param/code-rigid-rotation/figs/theta-0-q-3-r}
\includegraphics[width=0.48\textwidth]{/Users/maz/repos/perovskite/calc/analytical-structure-param/code-rigid-rotation/figs/theta-0-q-3-r}
\caption{append fig; combine in one plot: $D_2/D_4\simeq 14$ at 
$a=7$ bohr and $2eV$ crystal field gap ($\braket{r^4}=11.5809$, $D_4\simeq0.005$, $D_2=\sqrt{D_4}\simeq0.07$.)
}
\label{default}
\end{center}
\end{figure}
}



}


\end{document}
%%%%%%%%%%%%%%%%%%%%%%%%%%%%%

\section{System and Model}
System: perovskites ABO3;
Full model = lattice;
Toy model = cages;

\section{Analytical Results}
phi rotations; 2x2 solutions;\\
theta/phi rotations; approx 2x2 
\subsection{Violation of Jahn-Teller Theorem}
\section{Numerical Results}
Exact numerics cages [cubic];;; exact lattice [with a,b,c lattice constants]; 
\section{Effect of SOC}








\begin{figure}
\centering
%\includegraphics[width=0.5\linewidth]{cartoon}
\caption{
}
\label{fig:cartoon}
\end{figure}

 %\end{document}
\label{Bibliography}
\bibliographystyle{apsrev4-1}
%\bibliography{References}

\end{document}








\com{
$V_{oct}(\vec r)$ causes the d-orbitals to split into two manifolds,
$t_{2g}$ ($xy,yz,zx$) and $e_g$ ($z^2,x^2-y^2$)
by an amount
${\Delta_{CF}=\frac{160}{3}\frac{q_o}{a} D_{4}}$.
The deformation and resulting $V_{dipoles}(\vec r)$
will lift the degeneracies as follows.
\begin{align}
(dE_{xy},dE_{yz},dE_{zx})&= \Delta_{\delta}
\left[(2, -1, -1)D_{2} +\frac{20}{9} 
 (-1, 4, 4) D_{4}\right],\\
(dE_{x^2-y^2},dE_{z^2})&=\Delta_{\delta}
\left[(2,-2)D_{2} +\frac{20}{9} 
 (-1,-6)D_{4}\right],\\
\Delta_{\delta}&= \frac{96}{7}
\frac{-q_o}{a}
\frac{\delta}{a}.
\end{align}
}










\begin{align}
u&=\frac{1}{4} (2+\cos\beta+\cos\alpha \cos\beta),\\
v&=\frac{1}{4}
\cos\beta(1-\cos\alpha),\\
w&=\frac{1}{2} (1+\cos\alpha),\\
x&=-1+\frac{1}{2} (1+\cos\alpha) \cos\beta,\\
\{x\to 2 (u-1),y\to 2 v,z\to 2 (w-1)\},\\
\{d\to -1+2 \sigma ,e\to -1+2 \tau ,f\to 1+2 \mu ,g\to 1-2 \tau ,h\to 1-2 \sigma \}
%y&=2v=\cos\beta \frac{1}{2} (1-\cos\alpha),\\
%z&=2(w-1)=-1+\cos\alpha.
\end{align}
we obtain the following expressions for $V_{l,m}^B$.
\begin{align}
V_{2,-2}^B &=8\sqrt{\frac{3 \pi }{5}} \frac{v \left(2 u^3-4 u^2 x-8 u v^2+v^2 x\right)}{\left(u^2+v^2\right)^{7/2}},\\
V_{2,0}^B &=12 \sqrt{\frac{ \pi }{5}} \left(\frac{u x+2 v^2}{\left(u^2+v^2\right)^{5/2}}-\frac{2(w-1)}{w^4}\right),
\end{align}
where all other $V_{2,m}^B= 0$.
Similarly,
$l=4$ components are,
\begin{align}
V_{4,4}^B&= -\frac{\sqrt{35 \pi }}{3}\frac{\left(5 u^5 x+42 u^4 v^2-46 u^3 v^2 x-92 u^2 v^4+21 u v^4 x+10 v^6\right)}{\left(u^2+v^2\right)^{11/2}},\\
V_{4,-2}^B&= -\frac{4\sqrt{5\pi}}{3}\frac{ v \left(-2 u^3+6 u^2 x+12 u v^2-v^2 x\right)}{\left(u^2+v^2\right)^{9/2}},\\
V_{4,0}^B&=-\frac{5\sqrt{\pi }}{3}  \left(\frac{3 \left(u x+2 v^2\right)}{\left(u^2+v^2\right)^{7/2}}+\frac{8(w-1)}{w^6}\right),
\end{align}







\com{
\begin{align}
\tilde V(\vec r) &=\tilde V_O(\vec r) + \tilde V_{AB}(\vec r),\\
\tilde V_O(\vec r) &= \tilde V_{oct}(\vec r),\\
\tilde V_{AB}(\vec r) &= \mathcal{R}(\alpha,\beta) V_{AB}(\vec r),\\
V_{AB}(\vec r) &= f(q_A) V_{oct}(\vec r) + 
V_{dipoles}^{AB}(\vec r;\alpha,\beta),\\
\label{eq:ylm_expansion}
V_{dipoles}^{AB}(\vec r;\alpha,\beta) &= 
\sum_{l=2,4}\sum_{m=-l}^{l}\left[V_{l,m}^{A}(\alpha,\beta)+V_{l,m}^{B}(\alpha,\beta)\right] r^l Y_{l,m}(\hat r),\\
\tilde V(\vec r) &= \sum_{l,m} \tilde V_{l,m} r^l \tilde Y_{l,m}(\hat r),\\\end{align}
\az{fix inconsistencies... in above equations.. R matix apply on a vector,,,, not a function.}.
}
