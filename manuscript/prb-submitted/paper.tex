
\documentclass[a4paper,prb,twocolumn,longbibliography]{revtex4-1}  %a4paper,prb,twocolumn
%\documentclass[preprint]{revtex4-1}

%\documentclass[12pt]{report}

%\usepackage{orcidlink}

\usepackage{graphicx}  % needed for figures
\usepackage{dcolumn}   % needed for some tables
\usepackage{bm}        % for math
\usepackage{amssymb}   % for math
\usepackage{hyperref}

\usepackage{soul,color}

\usepackage{caption} %For two images side by side
\usepackage{subcaption} %For two images side by side

\usepackage{amsfonts}

\usepackage{amsmath}
%\usepackage{kbordermatrix}
\usepackage{blkarray}
\usepackage{braket}
\usepackage{multirow}
\usepackage{mathtools}

\usepackage{enumerate}

% for figure (Cartoon)
\usepackage{tikz}
\usepackage[utf8]{inputenc}

\newcolumntype{L}{>{$}l<{$}} % math-mode version of "l" column type


\newcommand{\y}{\lambda}
\newcommand{\yp}{\lambda^\prime}
\newcommand{\ph}{\Phi}
\newcommand{\ps}{\Psi}
\newcommand{\w}{E}
\newcommand{\cw}{\Omega}
\newcommand{\dl}{\delta}
\newcommand{\x}{\times}
\newcommand{\llra}{\Longleftrightarrow}
\newcommand{\ep}{\varepsilon}
\newcommand{\sig}{\sigma}
\newcommand{\al}{\alpha}
\newcommand{\be}{\beta}

\newcommand{\eps}{\epsilon}
\newcommand{\wrr}{\Omega_R}

\newcommand{\sx}{\mathcal{\hat S}_0}
\newcommand{\sy}{\mathcal{\hat S}_k}


\newcommand{\N}{\mathcal{N}}
\newcommand{\nb}{N}
\newcommand{\nd}{N_D}
\newcommand{\Nd}{\mathcal{N}_D}
\newcommand{\Nb}{\mathcal{N}_B}



\newcommand{\stt}{\hat{\mathcal{S}}}
\newcommand{\stf}{\hat{\tilde{\mathcal{S}}}}

\newcommand{\ws}{{\omega^\prime}}

\newcommand{\ev}{\hat{\Psi}^+(\w)}



\newcommand{\stket}[1]{{\mathcal{S}_{#1}}}

\newcommand{\nex}{\hat{\mathcal{N}}_{ex}}

\newcommand{\h}{\mathcal{\hat H}}
\newcommand{\z}{\mathcal{Z}_{\alpha}}
\newcommand{\hi}{\hat h_{\alpha}}
\newcommand{\hj}{\mathcal{\hat H}_j}

\newcommand{\hd}{\mathcal{\hat H}_{\Delta}}

\newcommand{\com}[1]{}

%\DeclareMathOperator{\Tr}{Tr}
\DeclareMathOperator{\diag}{diag} 
\DeclareMathOperator{\integer}{integer}
%\DeclareMathOperator{\nd}{and}
\DeclareMathOperator{\eV}{eV}
%\everymath{\displaystyle}

\newcommand{\ad}{\hat{a}^\dagger}
\newcommand{\an}{\hat{a}^{}}
\newcommand{\sd}{\hat{\sigma}^{+}_i}
\newcommand{\sn}{\hat{\sigma}^{-}_i}

\newcommand{\ua}{\uparrow}

\newcommand{\fd}{f(\hat{b}^\dagger)}
\newcommand{\fn}{f(\hat{b}^{})}


\newcommand{\mbf}[1]{\mathbf{#1}}
\newcommand{\del}{\hat{\Delta}^{}}



\newcommand{\fref}[1]{Fig.~(\ref{#1})}
\newcommand{\tref}[1]{Table~\ref{#1}}
\newcommand{\eref}[1]{Eq.~(\ref{#1})}

\newcommand{\rev}[1]{{\color{blue}{#1}}}  
\newcommand{\az}[1]{{\color{magenta}{#1}}} %Ahsan Zeb
\newcommand{\ha}[1]{{\color{red}{#1}}} 
\newcommand{\sm}[1]{{\color{brown}{#1}}} 

\newcommand{\Tau}{\mathrm{T}}

\DeclareMathOperator\erf{erf}
\DeclareMathOperator\erfi{erfi}
\DeclareMathOperator\ArcTanh{ArcTanh}
\DeclareMathOperator\sgn{sign}

\newcommand{\ylm}[1]{Y_{#1}(\hat r)} 


\def\Xint#1{\mathchoice
   {\XXint\displaystyle\textstyle{#1}}%
   {\XXint\textstyle\scriptstyle{#1}}%
   {\XXint\scriptstyle\scriptscriptstyle{#1}}%
   {\XXint\scriptscriptstyle\scriptscriptstyle{#1}}%
   \!\int}
\def\XXint#1#2#3{{\setbox0=\hbox{$#1{#2#3}{\int}$}
     \vcenter{\hbox{$#2#3$}}\kern-.5\wd0}}
\def\ddashint{\Xint=}
\def\dashint{\Xint -} % \sim


\newcommand\scalemath[2]{\scalebox{#1}{\mbox{\ensuremath{\displaystyle #2}}}}

%\usepackage[paperheight=18cm,paperwidth=14cm,textwidth=12cm]{geometry}
%\usepackage{setspace}
%\doublespacing

\begin{document}

\title{Jahn-Teller effect with rigid octahedral rotations in perovskites} %Cooperative 

%\title{Orbital ordering in perovskites transition metal oxides with rigid octahedral rotations} 

\author{Rukhshanda Naheed}
\affiliation{Department of Physics, Quaid-i-Azam University, Islamabad 45320, Pakistan}

\author{M. Ahsan Zeb}
\affiliation{Department of Physics, Quaid-i-Azam University, Islamabad 45320, Pakistan}
\email{ahsan.zeb@hotmail.com}
%\orcidlink{https://orcid.org/0000-0002-3448-2357}

\author{Kashif Sabeeh}
\affiliation{Department of Physics, Quaid-i-Azam University, Islamabad 45320, Pakistan}


\date{\today}
\begin{abstract}

We consider
rigid rotation and tilting of the 
%BO$_{6/2}$ 
anion octahedra around the transition metal ions
 in ABO$_3$ perovskites
 from the viewpoint of Jahn-Teller problem
 and study the
effect of the crystal field of the nearest A and B ion cages 
on the single electron spectrum of the 
central 
transition metal B ions.
While the crystal field of the octahedra,
which
% quenches the orbital angular momentum of the central B ion and 
creates degenerate manifolds $e_g$ and $t_{2g}$,
do not change with such rigid rotations, 
the field of 
%A and B ionic cages %that immediately surround the octahedra
A and B cages
deforms significantly %with the octahedral rotations 
from the ideal octahedral field
in a way that
leads to sizeable splittings in 
%the $e_g$ and $t_{2g}$ manifolds of the central B ion.
these manifolds.
%Similar to the Jahn-Teller problem,
The lowering of the ground state energy
of a given many-electron configuration
due to these splittings
can thus be an important driving force behind the octahedral rotations 
and associated structural transformations.
We find that the size of the splitting and the 
%local orbital ordering 
order of the orbitals
are
determined by a competition between the A and B cages, 
as well as a competition between the tilt and rotation angles.

\end{abstract}
\maketitle


Perovskites  
exhibit a rich variety of intriguing 
electronic,
magnetic
 and optical
properties~\cite{ZubkoARCMP11,
HwangNM12, 
BhattacharyaARMR14,
HellmanRMP17,
ChenJPCM17,DagottoS05,DagottoMRSB08}
due to the interplay between
spin, orbital, and structural degrees of 
freedom
and widely varying electronic 
correlations.
These materials 
are important
both for practical applications in all sorts of modern devices 
and 
fundamental understanding of various physical 
phenomena~\cite{SalamonRMP01,
WangS03,
DawberRMP05,
SchlomARMR07,
KosterRMP12}.


%Their crystal structures 
The perovskite structures
have a common formula unit ABX$_3$,
where 
%positive ions A and negative ions X form a cube and an octahedron around positive ion B.
A and B are positive ions while X are negative ions 
%whose six-fold coordination forms octahedra around B.
that form octahedra around  B.
In a cubic perovskite,
the octahedra are symmetric, 
but they can 
distort~\cite{
BersukerCCR75,
KugelSPU82,
MillisN98,
TerakuraPMS07,
HalcrowRSC13,
LufasoAC04}
or simply 
rotate~\cite{
GlazerAC72, ThomasAC96,
WoodwardAC97a,WoodwardAC97b,
HowardAC98,StokesAC02,AngelPRL05}
 to lower the symmetry
and produce, e.g., a rhombohedral, tetragonal or orthorhombic structure.
% cubic perovskite [spacegroup 221] can transform to:
% crystal system: spacegroups
% triclinic: 2
% monoclinic: 11,12,15
% orthorhombic: 59,62,63,71,74
% tetragonal 127,139,140
% rhombohedral/trigonal: 167
% cubic: 204, 221 
Octahedral rotations also occur in other perovskite related structures
and
have been extensively 
studied
due to their crucial role in determining various physical properties,
e.g., in 
ferroelectric~\cite{GhosezNM11,
RondinelliAM12,
SchickPRB14,
SimPRB14,
LeePRL17,
WangPRL22,
ZhouPRB22,
GaoPRB23} and 
multiferroic~\cite{KhomskiiPhys09,
BenedekPRL11,
YangPRL14,
YePRB18,
ZhangPRL20,
ZhangYajunPRL20, 
ZhouPRB21,
ZhangPRL22}
systems.
The interplay of octahedral rotations
and 
spin-orbit coupling~\cite{AmatNL14,KrachJPCM23},
breathing distortions~\cite{BalachandranPRB13},
strain~\cite{Aguado-PuentePRL11},
defects~\cite{JiaPRB22}, and
pressure~\cite{XiangPRB17},
and 
their role in
magnetostructural phase transitions~\cite{JohnsonPRL20},
orbital order~\cite{OgawaPRL12,ChenPRL19}, orbital anisotropy~\cite{HuangPRB21}, 
and
%surface electronic structure under in-plane uniaxial compression~\cite{MoralesPRL23}
surface electronic structure~\cite{MoralesPRL23},
have also been investigated.
% JT interacting with octa tilting:
Since, usually, the octahedra are not just %(Jahn-Teller) 
distorted or rotated but a combination of both,
the influence of these two 
structural mutations %instabilities 
on each other has recently been explored by many authors~\cite{LufasoAC04,CarpenterAC09,ChenPRL19}
as well.



While the tetragonal distortion of the octahedra is associated with
a (cooperative) Jahn-Teller 
effect~\cite{JahnPRSL37,JahnPRSL38,
OpikRS57,
Sturge68,
GehringRPP75,
BersukerCCR75,
KugelSPU82,
MillisN98,
LufasoAC04,
TerakuraPMS07,
HalcrowRSC13,
PavariniChap,
KhomskiiCR21}
(i.e., the orbital degeneracy of the metal ion coordinated by the anionic octahedron
is lifted, which stabilises the distortion),
%[as it distorts the octahedral field at the central site that split its electronic states],
the rotation and tilting of the octahedra are almost always 
explained empirically in terms of 
the relative sizes of the three ions
with the
Goldschmidt tolerance factor~\cite{Goldschmidt1926,KieslichCS15}. 
%[probably because rigid rotation of the octahedra do not change its potential at the central site]
%We only find two exceptions. 



\begin{figure}[htbp]
\begin{center}
%\includegraphics[width=1\linewidth]{/Users/maz/repos/perovskite/calc/analytical-structure-param/fig-data/struc/structures-combined-bonds-labelled-2}
\includegraphics[width=1\linewidth]{structures-combined-bonds-labelled-2}
\caption{
Crystal structures and coordination cages.
Crystal structures of cubic (a) and orthorhombic (b) perovskites.
The anions (red) form the shown octahedra around 
the B cations (gold), while the A cations (green) sit in the voids between the octahedra.
In the orthorhombic structure shown here (b),
 the octahedra are rotated and tilted by $15^\circ$.
(c-d)
The coordination cages of A %(green) 
and B 
%(golden) 
cations just next to the anion 
%(red) 
octahedron
around a B ion in cubic %(c)
 and orthorhombic %(d)
  perovskites.
In the anion octahedron's frame,
the cages are symmetric and aligned to the octahedron in the cubic perovskites (c),
while they are rotated and slightly distorted in the orthorhombic perovskites (d).
}
\label{fig:struc}
\end{center}
\end{figure}



Based on density functional theory (DFT) calculations,
a few recent works
~\cite{garcia-fernandezJPCL10,CammarataJCP14,LeeCM16,YoshidaPRL21}, 
however, 
suggest that the octahedral rotation and tilting is 
a second order Jahn-Teller effect
~\cite{HerzbergZPC33,BersukerPL66,KristoffelPSSB67,PearsonJMS83,FultonJCP04,BersukerCR13}
%\az{describe SOJT in a sentence: coupling to the low lying unoccupied states on the JT ion, or orbital hybridisation with other ions, coupling between bands...}
(i.e., the distortion is stabilised by a coupling between the ground and excited adiabatic potential energy surfaces).
% garcia-fernandezJPCL2010
Garcia-Fernandez~et~al.
study
potassium fluoride perovskite family KMF$_3$ (M$=$Ca$^{2+}$ or a $3$d transition metal)
and found that the tilting angle
depends on the covalent bonding through the mixing of occupied and 
low-lying unoccupied orbitals of the transition metal~\cite{garcia-fernandezJPCL10}.
% CammarataJCP2014
Cammarata and Rondinelli
find a negative correlation between metal-oxygen bond covalency 
and octahedral rotation in orthorhombic perovskite oxides~\cite{CammarataJCP14}.
% Lee, nick2016
Lee~et~al. find 
similar results for metal-iodine bonding in
lead iodide perovskites,
where the role of 
the hydrogen bonding 
in a hybrid perovskite with
organic ammonium ions at A sites
is also found to be crucial~\cite{LeeCM16}.
Yoshida~et~al.
use DFT along with 
representation theory analysis 
and 
indicate that 
the rotations are induced by bonding interaction 
between the valence and conduction band Bloch states at different wavevectors
that are mapped onto each other on Brillouin zone folding
 for the associated lattice distortion~\cite{YoshidaPRL21}.
 


We take a completely different approach to this problem.
We go back to the basics
and see 
what is it that lifts 
 the orbital degeneracy in the classical JT effect?
It is a specific 
deformation of the crystal field 
%(${\sim z^2}$)
at the transition metal (JT ion) site.
Does the same happen in the case of octahedral tilting and rotations?
While the potential of an octahedral anion cage 
in its own frame
does not change 
%in case of 
with
rigid rotations for obvious reason,
the total crystal field 
does deform %indeed
because
 the rest of the lattice is not only rotated
  in a given octahedron's frame (that turns its non-spherical potential)
 but also distorted owing to the accompanying structural transformation.
%So, in principle, there should be a (first order) JT effect at play.
Using this insight
and elements of crystal field theory~\cite{BetheAP29,VanVleckPR32,PavariniChap},
%here in this manuscript,
we 
consider next two coordination cages around an octahedron
% explore this problem in detail
and find that,
just like a first order JT effect,
 the deformed crystal field 
in this case also lifts the orbital degeneracy,
by a significant amount for typical system parameters,
signalling the instability against the rotations.

% cooperative: pairs of alpha beta....
% terminnology: rotation vs tilt
% coop/consistent rotations/tilting then makes Pbnm etc...

Since the octahedra are connected to each other 
(making a 3-dimensional network in the perovskite structure),
rotating one octahedron rotates all other in the crystal in a consistent fashion~\cite{GlazerAC72};
 see Figs.~\ref{fig:struc}(a,b).
We consider an octahedral \emph{rotation} $\beta$ around $001$ direction
(that transforms the structure from cubic Pm$\bar 3$m to tetragonal P4/mbm)
%({No.~$127$})
followed by a \emph{tilting} $\alpha$ around the new $110$ direction
(transforming to orthorhombic Pbnm, 
%({No.~$62$})
or Imma 
%({No.~$74$})
 at ${\beta=0}$).
We use the rotation and tilt terminology for these 
two angles throughout the rest of the manuscript.


Beyond the anion octahedron,
each transition metal ion B in a cubic perovskite
is coordinated by two cation cages,
a cube of A and an octahedron of B, as shown in Figs.~\ref{fig:struc}(c,d).
Including the 
%electrostatic 
potential of these two ion cages
into the 
crystal field Hamiltonian of 
the central ion B,
we explore the effect 
of the rotation and tilting on 
%the splittings in 
its $t_{2g}$ and $e_g$ manifolds
that are
created primarily by the anion's octahedral potential.
We find that the degeneracy of $t_{2g}$ and $e_g$
is lifted. 
The size of the splitting 
and 
%local orbital ordering
the order of the orbitals
depend
on the relative charges on the two types of cations
and a competition between rotation and tilting.
At typical parameter values,
the splitting is large enough 
to play a major role in 
the corresponding structural instability. 
The qualitative picture developed by our calculations 
holds even when the potential of the full lattice is considered.


% 3 dim hilbert space of t2g, 2 dim hipbert space of eg
% the mixing of t2g states due to rot and tilting is equiv to an artifact in cubic phase
% so it's not similar to SOCJT states at diff energy mix due to distortion.
% exlain it in the results/discussion section.....???

\com{
$\alpha=0, \beta \neq 0$\\
Tetragonal:
127. P 4 / m b m
$\alpha \neq 0,\beta=0$\\
Orthorhombic:
74. I m m a
$\alpha,\beta \neq 0$\\
Orthorhombic:
62. Pbnm
}









The paper is organised as follows.
Section~\ref{sec:formalistm}
presents the formalism and 
Sec.~\ref{sec:JT} reviews the Jahn-Teller effect to
introduce the notation and concepts involved
by considering tetragonal distortion of an octahedron.
%\az{[.... multiple subsections?]}
The model and calculations are presented
in Sec.~\ref{sec:calc},
where
we first discuss the relative contributions of the three closest ion cages
in the cubic structure
 to
the crystal field at the central transition metal site 
and crystal field splitting between its $e_g$ and $t_{2g}$ manifolds
(Sec.~\ref{sec:relativeVoctABO}),
and then 
describe the deformation effects of 
rigid octahedral rotations %(Sec.~\ref{sec:rotation})
on
the structure (Sec.~\ref{sec:structure}) 
and crystal field (Sec.~\ref{sec:deformCF}).
The results are in Sec.~\ref{sec:results},
where we %seperately
present the splitting in the single particle spectrum
belonging to the $t_{2g}$ (Sec.~\ref{sec:t2g})
and $e_g$ (Sec.~\ref{sec:eg-gen}) manifolds,
and finally describe the effect of the full lattice (Sec.~\ref{sec:FL}).
%\az{appendix?}







 
\section{Formalism: Crystal field Hamiltonian using multipole expansion of the crystal field potential}
\label{sec:formalistm}

Let us introduce the basic formalism we use throughout this manuscript.
The spectrum of an atom or ion
changes in the presence of nearby charged species.
This is described by the crystal field Hamiltonian of the atom
that contains the coupling between its atomic orbitals and/or their energy shifts
induced by the electrostatic potential of those charges.
Since the angular part of atomic orbitals are described by 
the spherical harmonics,
an expansion of the potential at the atom's site 
in the same bases simplifies the analysis.
This is because
the effect of various multipole components of the potential
is governed by certain selection rules 
(enforced by the symmetries of the spherical harmonics
through Gaunt coefficients, see below).



If we have an arrangement of ions with charge $q$ around a given site,
we can express their %(crystal field) 
potential in terms of spherical harmonics~\cite{PavariniChap}, 
\begin{align}
V(\vec r) &= \frac{q}{a}\sum_{lm}V_{lm}~\left(\frac{r}{a}\right)^l \ylm{lm},
\end{align}
where $a$ sets the length scale, $\hat r=\vec r/r$ is the direction of $\vec r$,
and $V_{l,m}$ are multipole components of the potential
 (for the given site as origin and a chosen orientation).
%and calculate the crystal field hamiltonian of the central ion/atom.
Assuming we have a transition metal atom/ion
at the central site
with
only $d$-orbitals ($l=2$),
the matrix elements of its crystal field Hamiltonian
are then given by,
\begin{align}
H_{m' m''} &= -\frac{q}{a}\sum_{l,m} V_{lm} C_{l}^{m'm''m} D_{l} ,
\end{align}
where 
$C_{l}^{m'm''m}=\braket{Y_{2m'}| Y_{lm}| Y_{2m''}}$ are Gaunt coefficients for the given (real or complex) spherical harmonics,
and $D_{l}=\braket{(r/a)^l}$
is the expectation of $(r/a)^l$ evaluated in the $d$-orbitals bases
assuming the same radial dependence for all orbitals~\cite{PaxtonNotes}.
The sum is restricted to even values of $l\leq 4$ as 
$C_{l}^{m'm''m}=0$ otherwise,
with $-l \leq m \leq l$.


We consider transition metal oxides as an example perovskite system throughout the rest of the manuscript,
but our calculations and results are general and also apply to perovskite transition metal compounds with other anions, e.g., halides and chalcogenides.

 
\subsection{Atom/ion in an octahedral field: $t_{2g}$  and $e_g$ manifolds}
\label{sec:egt2g}
Consider six oxygen (O) ions each with a charge $-q_o$ making an octahedron of size $a$ (where these anions are placed at the face centres of a cube of edge length $a$).
The non-spherical part of the potential they
 produce at the centre of the octahedron
 %that is 
 relevant for the $d$-orbitals contains 
 only two $l=4$ components, 
  in a specific ratio (${V_{4,0}:V_{4,4}=1:\sqrt{5/7}}$),
 and is called octahedral field. 
It is given by~\cite{PavariniChap},
\begin{align}
\label{eq:voct}
V_{oct}(\vec r) = -\frac{224\sqrt{\pi}}{3}\frac{q_o}{a} \left(\frac{r}{a}\right)^4\left[\ylm{4,0}+\sqrt{\frac{5}{7}}\ylm{4,4} \right].
\end{align}
%where the monopole and $l>4$ components have been dropped
The crystal field Hamiltonian of a transition metal ion B 
%at the centre 
in the field $V_{oct}(\vec r)$
turns out to be diagonal,
with its states split into two manifolds: 
% $\Delta$ apart:
$H_{mm} = -\Delta/3$ for $m\in\{-2,-1,1\}$ ($xy,yz,zx$ orbitals)
called $t_{2g}$,
and 
$H_{mm} = +2\Delta/3$ for $m\in\{2,0\}$ ($x^2-y^2,z^2$ orbitals)
called $e_g$.
The crystal field splitting between the two manifolds is
${\Delta=\frac{160}{3}\frac{q_o}{a} D_{4}}$.


\com{
$V_{oct}(\vec r)$ causes the d-orbitals to split into two manifolds,
$t_{2g}$  and $e_g$, where $t_{2g}$ contains three degenerate states ($xy,yz,zx$) while $e_g$ has two degenerate states ($x^2-y^2,z^2$).
The crystal field splitting between the two manifolds is
${\Delta=\frac{160}{3}\frac{q_o}{a} D_{4}}$.
\az{We will see how these degeneracies are lifted as we elongate (or compress)
the octahedron along a ....??}
}


\section{Jahn-Teller Effect}
\label{sec:JT}



The Jahn--Teller 
effect~\cite{JahnPRSL37,JahnPRSL38,
OpikRS57,
Sturge68,
GehringRPP75,
BersukerCCR75,
KugelSPU82,
MillisN98,
LufasoAC04,
TerakuraPMS07,
HalcrowRSC13,
PavariniChap,
KhomskiiCR21}
 describes the 
geometric and electronic instability of a %non-linear 
molecule, cluster or crystal with
a symmetric structure and orbitally degenerate ground state
%(i.e., except Kramers degeneracy)
against structural distortions.
The distortions  
reduce the symmetry of the structure
and lift the electronic degeneracy.
This lowers the energy of the system proportional to
the distortion.
The elastic energy of the structure that increases as the square of the distortion
competes with this linear term and the optimum structure 
is obtained at a finite distortion, as described below.

%\subsection{Dependence of electronic and elastic energies on the distortion}

The dependence of the elastic energy on the distortion can be roughly explained
as follows.
Consider an 
elongation or a compression of an octahedron along one of its axes by $a\delta$,
where $a$ is the size of the octahedron and lattice constant of the cubic perovskite structure.
The corresponding change in the unit cell volume would be linear in $\delta$
as $\sim a^3 \delta$
so the elastic energy
would change as $\sim\delta^2$.
Assuming a linear change 
 in the electronic spectrum,
  the total energy has a form
$E=-\rho\delta + K \delta^2$, where $\rho$ and $K$ are positive constants 
describing the changes in the electronic and elastic energies,
and would be optimum at $\delta=\rho/2K$.
In general, as long as the two competing terms 
are not the same degree monomial in the distortion,
the ground state can have a finite distortion.
We use this argument to focus only on the changes
 in the electronic energy throughout this paper.




\com{
The dependence of the elastic energy on the distortion can be roughly explained
as follows.
Consider a 
hypothetical 
distortion of an octahedron 
along two of its axes, say, by $a\delta_1,a\delta_2$,
where $a$ is the lattice constant of the cubic structure.
The corresponding change in the unit cell volume would be linear in $\delta_{1,2}$
as $\sim a^3\left(\delta_1+\delta_1\right)$
so the elastic energy %(for an isotropic system) 
changes as $\sim\left(\delta_1+\delta_1\right)^2$.
A linear change ($\propto \delta_{1/2}$) in the electronic spectrum is required to produce
 a finite distortion in the optimum structure (or the ground state of full electron-lattice/nuclear problem).
}

It is worth reminding that, in the Jahn-Teller problem,
we usually consider single particle spectrum, i.e.,
single electron energy states,
to analyse the ground state of a many electron system,
which is the lowest energy ``electronic configuration'' 
(a combination of occupied single particle states 
out of all available single particle states, 
or a slater determinant of that combination to be more precise!).
So a degenerate set of states that are completely occupied
makes a non-degenerate many electron state because 
only a single configuration/combination is possible.
For a given number of electrons in the d-shell of the transition metal ion,
we can determine the (most likely) configuration(s) or ground state(s)
using Hund's rule and see if it is degenerate or otherwise.
  



\com{
Instability:\\
 Low/high spin configurations, rules/arguments using single particle spectrum and a given low/high spin state with a given number of electrons:\\
elastic energy vs electronic... number of electrons... multi-electron configurations.... degenerate vs non-degenerate cases...
}






In the following, we 
consider an octahedral environment around a transition metal ion 
and 
illustrate how the changes in the electronic energy 
can be calculated
 when the octahedron is elongated.











\subsection{Tetragonal distortion of an octahedron}
\label{sec:JTtertagonal}

Let us consider elongation of an octahedron % in Sec.~\ref{sec:egt2g} 
along one of its O-B-O bonds. 
Align the cartesian axes along O-B-O bonds and stretch the two bonds along $z$-axis equally by an amount $a\delta$.
%where $a$ is the size of the octahedron (the distance between the O-O along any cartesian axes).
The potential $V^{JT}(\vec r)$ at the central B site (JT ion) of this deformed octahedron 
still contains only a few non-zero components (see appendix for a method to calculate $V_{lm}^{JT}$).
Not only the ratio 
$V_{4,0}^{JT}:V_{4,4}^{JT}$ now deviates from
 ideal 
octahedral field in \eref{eq:voct},
there is an $l=2$ component induced due to the deformation,
\begin{align}
%V_{0,0}^{JT}=& \frac{8 \sqrt{\pi } (2 \delta +3)}{\delta +1},\\
V_{2,0}^{JT}=& -\frac{32 \sqrt{\frac{\pi }{5}} \delta  \left(\delta ^2+3 \delta +3\right)}{(\delta +1)^3},\\
V_{4,0}^{JT}=& \frac{32}{3} \sqrt{\pi } \left(\frac{4}{(\delta +1)^5}+3\right),\\
V_{4,4}^{JT}=& \frac{32 \sqrt{35 \pi }}{3}.
\end{align}

\com{
At $\delta=0$, keeping only the non-spherical components, 
we obtain the
 octahedral potential for a symmetric octahedron~\cite{PavariniChap},
\begin{align}
\label{eq:voct}
\lim_{\delta\to0}V^{JT}(\vec r) \to V_{oct}(\vec r) = -\frac{224\sqrt{\pi}}{3}\frac{q_o}{a} \left(\frac{r}{a}\right)^4\left[\ylm{4,0}+\sqrt{\frac{5}{7}}\ylm{4,4} \right],
\end{align}
}


The crystal field Hamiltonian of the JT ion,
${H_{m' m''}^{JT} = (q_o/a)\sum_{l,m} V_{lm}^{JT} C_{l}^{m'm''m} D_{l}}$,
still turns out to be diagonal 
(i.e., the deformation $\delta$
will not couple these orbitals to each other),
but the degeneracies of the two manifolds is lifted
and their respective average energies also shifted.
Defining 
$F_n(\delta) = 1-1/(1+\delta)^n$,
we see that
the $t_{2g}$ now splits into two levels,
with energies
\begin{align}
E_{xy} =&
\frac{32}{7}\frac{q_o}{a} \left(
 D_2F_3 - \frac{20}{9}D_4F_5 
\right),\\
E_{yz} =& E_{zx} = -\frac{1}{2} E_{xy},
\end{align}
as measured from their average 
$E_0^{t_{2g}} = -64q_oD_4 \left(3-F_5\right)/9a$.
Similarly, $e_g$ splits up as
\begin{align}
E_{x^2-y^2}=&
\frac{32}{7}\frac{q_o}{a} \left(D_2F_3+\frac{5}{3} D_4F_5\right),\\
E_{z^2}=&-E_{x^2-y^2},\\
\end{align}
as measured from their average 
$E_0^{e_{g}} = 32 q_oD_4 \left(3-F_5\right)/3a$.
The crystal field splitting becomes
$\Delta \equiv E_0^{e_{g}}-E_0^{t_{2g}}=
160q_oD_4\left(3-F_5\right)/9a$.
At small $\delta$, 
expanding 
$F_3,F_5$
 up to first order in $\delta$,
we obtain
\begin{align}
\label{eq:JTt2g}
(E_{xy},E_{yz},E_{zx})&= (2, -1, -1)
%\left[\frac{16}{63} (27 \text{D2}-100 \text{D4})\right]
\left[D_2-\frac{100}{27}  D_4\right]\frac{48}{7}\frac{q_o}{a}\delta,\\
\label{eq:JTeg}
(E_{x^2-y^2},E_{z^2})&=(1,-1)
\left[D_2+\frac{25}{9} D_4\right]\frac{96}{7}\frac{q_o}{a}\delta,
\end{align}
whereas
%${\Delta_{\delta}= \frac{48}{7}\frac{q_o}{a}\frac{\delta}{a}}$.
$E_0^{t_{2g}}$, $E_0^{e_{g}}$ and $\Delta$ all are rescaled by a factor of ${1-5\delta/3}$.


\com{
The crystal field splitting becomes
\begin{align}
E_0^{t_{2g}},E_0^{e_{g}},\Delta \to (1-\frac{5}{3}\delta)E_0^{t_{2g}}
E_0^{e_{g}} \to (1-\frac{5}{3}\delta) E_0^{e_{g}}
\Delta \to (1-\frac{5}{3}\delta)\Delta
\end{align}
}


%\subsection{Parameters $D_2$ and $D_4$ of the JT ion/atom}
The terms in square brackets on the right of Eqs.~\ref{eq:JTt2g},\ref{eq:JTeg}
contain $D_2$ and $D_4$. 
$D_4$ is proportional to the crystal field splitting $\Delta$ 
($D_4=\frac{3a}{160q_o} \Delta$) and 
we can estimate its value from realistic values of $\Delta$ for a given octahedron of size $a$ and ion charge $-q_o$. 
Assuming $a=7.0$ bohr and $q_o=2e$ ($e$ being the elementary charge),
$D_4$ ranges from $1 \times 10^{-3}$ to  $7 \times 10^{-3}$
for $\Delta \simeq 0.5-3$~eV.
Typically, ${D_{2}\approx \sqrt{D_{4}}}$,
so $D_{2}/D_{4} \approx 1/\sqrt{D_{4}}$
and it roughly ranges between $29$ and $12$ for $D_4$ in the above range.
This means that, for realistic systems, 
we should always have $D_{2}/D_{4} > 100/27, 25/9 \approx 4$
so 
$D_{2}$ terms in 
Eqs.~\ref{eq:JTt2g},\ref{eq:JTeg}
dominate and determine the sign and size of the splittings.


So far, we have calculated the changes in the single particle energy levels of the B ion. 
Whether the deformation would be energetically favourable 
or not will depend on the number of electrons occupying these levels,
as illustrated below.


%\subsection{Examples}
Let us consider %the following two cases.
%\begin{enumerate}
%\item
a JT/B ion with two electrons.
In the symmetric octahedron case, 
we will have a degenerate (many electron) ground state 
with both electron in the same spin state (Hund's rule) in the $t_{2g}$ manifold.
Distorting the structure $\delta>0$, we lower the ground state energy because 
we can now choose $yz,zx$ orbitals that have lower energy than the original $t_{2g}$ energy of the undistorted structure.
%\item
In case of a JT/B ion with three electrons, the situation is very different.
Assuming Hund's coupling is enough to still produce complete spin polarisation,
there is only one possible orbital configuration, one electron in each of $xy,yz,zx$ orbitals.
This non-degenerate (many electron) ground state
does not lower its energy if the structure is distorted 
because the gain due to $yz,zx$ is already balanced by the loss due to picking the higher energy $xy$ state.
%\end{enumerate}


 
%\subsection{cooperative JT and orbital order}
%\az{just a brief description... cooperative JT and orbital order}
To complete the discussion of the JT instability,
it is worth mentioning that
JT effect occurs in all kinds of systems---molecules, clusters and crystals.
In crystals, e.g., perovskites, it becomes a cooperative effect where the
elongation of one octahedron is accompanied by a compression of adjacent octahedron,
which minimizes the changes in the unit cell volume and hence the elastic energy cost of the transformation.
Such alternate elongations and compressions also lead to 
an interesting long range orbital ordering where alternate orbitals
are populated on the adjacent JT sites.
 
 
 
 



\com{
\begin{align}
E_{xy} =&
\frac{32}{7} \left[
\left(1-\frac{1}{(\delta +1)^3}\right) D_2
-
\left(1-\frac{1}{(\delta +1)^5}\right) \frac{20}{9}D_4
\right]\frac{q_o}{a},\\
E_{yz} =& E_{zx} = -3/2 E_{xy},
\end{align}
as measured from their average 
%\begin{align}
$E_0^{t_{2g}} = -64q_oD_4 \left(2+(\delta +1)^{-5}\right)/9a$.
%\frac{64}{9} \left(-\frac{1}{(\delta +1)^5}-2\right) \text{D4}
%\end{align}
Similarly, $e_g$ splits up as
\begin{align}
E_{x^2-y^2}=&\frac{32}{7} \left[\left(1-\frac{1}{(\delta +1)^3}\right) \text{D2}+\frac{5}{3}\left(1-\frac{1}{(\delta +1)^5}\right) \text{D4}\right]\frac{q_o}{a},\\
E_{z^2}=&-E_{x^2-y^2},
\end{align}
measured from their average 
%\begin{align}
$E_0^{e_{g}} = 32 q_oD_4 \left(2+(\delta +1)^{-5}\right)/3a$.
%\frac{32}{3} \left(\frac{1}{(\delta +1)^5}+2\right) \text{D4}
%\end{align}
The crystal field splitting becomes
\begin{align}
\Delta \equiv E_0^{e_{g}}-E_0^{t_{2g}}=\frac{160}{9} \left(\frac{1}{(\delta +1)^5}+2\right) D_4.
\end{align}
}








\section{Model and calculations of ABO$_3$ Perovskites}%}
\label{sec:calc}

We now turn to our model and calculations of the splittings in the 
$t_{2g}$ and $e_{g}$ manifolds due to the deformations 
in the crystal field that are produced as the octahedra rotate or tilt.
First, let us consider the cubic perovskite and see 
how much A and B cages contribute to the octahedral crystal field and associated splitting $\Delta$ between $t_{2g}$ and $e_{g}$ manifolds.


\subsection{Contribution of different coordination cages in a cubic perovskite}
\label{sec:relativeVoctABO}

In a cubic perovskite ABO$_3$,
each transition metal atom/ion B
is surrounded by six O$^{2-}$, each at a distance $a/2$,
 making an octahedron of size $a$ around it.
The potential of these ions $V_{O}(\vec r)= V_{oct}(\vec r)$
splits the degenerate d-orbitals
by an amount ${\Delta_o=160q_oD_{4}/3a}$
 into $t_{2g}$ and $e_g$.
However, this is not the only crystal field potential at the B site.
The coordination cages of next nearest neighbours, 
eight A atoms/cations all at the same distance $\sqrt{3}a/2$
make a cube around the B ion,
which also produces an octahedral field, given by 
\begin{align}
V_{A}(\vec r) = \frac{8}{81 \sqrt{3}}\frac{q_A}{q_o} V_{O}(\vec r).
\end{align}
Furthermore,
the nearest neighbouring B ions, six of them each at a distance $a$,
make an octahedron, with a potential 
\begin{align}
V_{B}(\vec r) = -\frac{1}{32}\frac{q_B}{q_o}V_{O}(\vec r).
\end{align}
It is interesting to note that 
$V_{A}(\vec r)$ and $V_{B}(\vec r)$ have opposite signs
and 
their combined effect 
$V_{AB}(\vec r) = V_{A}(\vec r)+ V_{B}(\vec r)$
depends on their relative charges.
Using the charge neutrality $q_B=3q_o-q_A$ (remember that we take anion charge to be $-q_o$),
we can write 
\begin{align}
V_{AB}(\vec r) &= f_{q_A}V_{O}(\vec r),\\
f_{q_A} &= \left(\frac{1}{32}+\frac{8}{81 \sqrt{3}}\right)\frac{q_A}{q_o} -\frac{3}{32},
\end{align}
which shows that ${V_{AB}(\vec r)}$ vanishes
at
$q_A/q_o=729/({256 \sqrt{3}+243}) {\simeq 1.062}$.
%where they cancel each other exactly.



\com{
\begin{align}
%V_{oct}(\vec r) &= \frac{224\sqrt{\pi}}{3}\frac{q_o}{a} \left(\frac{r}{a}\right)^4\left[\ylm{4,0}+\sqrt{\frac{5}{7}}\ylm{4,4} \right],\\
V(\vec r) &= V_{O}(\vec r) + V_{A}(\vec r) + V_{B}(\vec r),\\
V_{O}(\vec r) &= V_{oct}(\vec r),\\
V_{A}(\vec r) &= \left(\frac{4}{81 \sqrt{3}}q_A \right)V_{oct}(\vec r),\\
V_{B}(\vec r) &= -\left(\frac{q_B}{64}\right)V_{oct}(\vec r).
\end{align}
It is interesting to note that 
$V_{A}(\vec r)$ and $V_{B}(\vec r)$ have opposite signs
and 
their combined effect
depends on their relative charges and vanishes
as
${q_A\to 1458/({256 \sqrt{3}+243})\simeq 2.124}$
where they cancel each other exactly.

\begin{align}
%V_{oct}(\vec r) &= \frac{224\sqrt{\pi}}{3}\frac{q_o}{a} \left(\frac{r}{a}\right)^4\left[\ylm{4,0}+\sqrt{\frac{5}{7}}\ylm{4,4} \right],\\
V(\vec r) &= V_{O}(\vec r) + V_{AB}(\vec r),\\
V_{O}(\vec r) &= V_{oct}(\vec r),\\
V_{AB}(\vec r) &= f_{q_A}V_{oct}(\vec r),\\
f_{q_A} &= \left(\frac{1}{32}+\frac{8}{81 \sqrt{3}}\right)\frac{q_A}{q_o} -\frac{3}{32},
\end{align}
where we have use the charge neutrality $q_B=3q_o-q_A$ in the last equation.
It is interesting to note that 
$V_{A}(\vec r)$ and $V_{B}(\vec r)$ have opposite signs
and 
their combined effect
depends on their relative charges and vanishes
as
${q_A\to 1458/({256 \sqrt{3}+243})\simeq 2.124}$
where they cancel each other exactly.

}



Including the contribution of these AB cages to the crystal field,
we obtain
\begin{align}
%V(\vec r) &= V_{O}(\vec r) + V_{AB}(\vec r),\\
V(\vec r)  &= (1+f_{q_A}) V_{O}(\vec r),
\end{align}
whereas the 
 splitting becomes
%\begin{align}
$\Delta = (1+f_{q_A})\Delta_o$.
%\end{align}
%where ${\Delta_O=\frac{160}{3}\frac{q_o}{a} D_{4}}$.
We see that, relative to the O octahedron,
 the contribution of the
 two cation cages towards the total field splitting
 is relatively small at any $q_A$ but
 %still large enough to be ignored.
 still too large to be ignored.
It can be measured by $(\Delta-\Delta_o)/\Delta_o=f_{q_A}$,
which linearly changes from 
$-3/32\simeq -10\%$ at $q_A=0$
to $8/{27 \sqrt{3}}\simeq 17\%$ at $q_A=3q_o$. 
Typical values of $q_A$ relevant for real transition metal oxides
can be safely assumed to range between $0$ and $4$;  
$q_A=0$ for systems like tungsten oxide $WO_3$ that do not have any cations at the A sites,
and $q_A=4$ still contributing just above $8\%$.
We will see that,
as the anion octahedra rotate or tilt,
$V_{AB}(\vec r)$ rotate in its frame and deform as well.
The deformation also induces $l=2$ components, which couple to the 
$d$-orbitals of the B ion much more strongly. 
%due to a much larger $D_{l=2}$.






%\subsection{Rigid octahedral rotations}
%\label{sec:rotation}



\subsection{Rigid octahedral rotations: Transformation of the crystal structure}
\label{sec:structure}

\com{
Coordinates of the Oxygen atoms transform according to 
the rotation matrix $R$.
While the d-orbitals basis states attached to the octahedron transform according to
$\mathcal{R}$.
}

We consider a
$\sqrt{2}\times\sqrt{2}\times 2$ unit cell of the cubic structure that transforms
to the unit cell of the orthorhombic structure under octahedral rotation and tilting,
as shown in Fig.~\ref{fig:struc}.
Assuming 
%$\vec a_1,\vec a_2,\vec a_3$
$\textbf{a},\textbf{b},\textbf{c}$
 to be the lattice vectors,
two B atoms in a layer are situated at
%$(0,0,0)$, 
$\textbf{a}/2$, $\textbf{b}/2$,
%$(\textbf{a}  + \textbf{b} )/2$, 
while two A atoms in the same layer are at
% all coor shifted by a/2 to match to those in Fig.1 [vesta: unit cell transformation]:
%$(\textbf{a} + \textbf{c}/2)/2$, $(\textbf{b}+ \textbf{c}/2)/2$,
$\textbf{c}/4$, $(\textbf{a}+\textbf{b})/2 + \textbf{c}/4$.
Similar atoms in the second layer are simply shifted by 
$\textbf{c} /2$.
While the lattice vectors change under the rotation and tilting of the octahedra, 
these relations do not change.
%The B atom at the origin will be considered as the central JT atom.
The O atoms in the cubic structure lie half way between the B atoms and make octahedral cages around them.
Their positions in the deformed structure %(see appendix) 
are not directly required for our calculations and discussion.


We rotate the octahedra by an angle $\beta$ about $\hat z \equiv 001$
and then tilt it about the new $110$ by an angle $\alpha$~\cite{CarterPRB12}.
The four octahedra in the cell 
have to be rotated and tilted clockwise or counter clockwise
consistently. For example, %$\alpha  \{1,-1,-1,1\}$, $\beta  \{1,-1,1,-1\}$.
$(\alpha,\beta)$,
$(-\alpha,-\beta)$,
$(-\alpha,\beta)$,
$(\alpha,-\beta)$
for the octaherda labelled $1$ through $4$ in Fig.~\ref{fig:struc}(b).
The structure transforms from cubic to orthorhombic under these rotation and tilting of the octahedra.
%The rotation matrix $R$ for $(\alpha,\beta)$ case is given in the appendix~\ref{sec:appendix}. 
The lattice vectors of the deformed structure and the positions of A and B atoms can be calculated easily by calculating the coordinates of all six oxygen atoms that make the octahedra (by rotating and tilting the octahedra) for all four octahedra in the unit cell and 
comparing the coordinates of their shared oxygen atoms.
We obtain,
%\az{just rescaled, not rotated. write as cos * (1,1), etc?}
% lattice vectors:
\begin{align}
\left(
\begin{array}{c}
%\vec a_1 \\
%\vec a_2\\
%\vec a_3\\
\textbf{a}\\
\textbf{b}\\
\textbf{c}
\end{array}
\right)
= a
\left(
\begin{array}{ccc}
 \cos \beta  & \cos \beta  & 0 \\
 -\cos \alpha  \cos \beta  & \cos \alpha  \cos \beta  & 0 \\
 0 & 0 & 2 \cos \alpha  \\
\end{array}
\right)
\end{align}


\com{
\begin{align}
\vec a_1 & =  \cos\beta ( 1,1,0 ) ,\\
\vec a_2 & =  \cos\alpha  \cos\beta(-1,1,0 ),\\
\vec a_3 & =  \cos\alpha ( 0,0,2 ).
\end{align}
}





\subsection{Deformation of the crystal field}
\label{sec:deformCF}

%\subsubsection{Deformation of the crystal field due to AB cages}

\com{Before we present calculations and results, let's briefly
summarise the contribution of the AB cages towards the crystal field Hamiltonian
....}
 
There are two different ways the crystal field $V(\vec r)$
 is deformed from the ideal case $V(\vec r) \propto V_{oct}(\vec r)$
 when the octahedra are rigidly rotated. 
% to make an orthorhombic structure. 
\begin{enumerate}[(i)]
\item
%(i) 
The relative rotations of the AB and O frames due to
the rigid rotation and tilting of the oxygen octahedra removes the
choice of a set of d-orbitals where the effect of 
AB and O cages is the same. 
We no longer just get $e_g, t_{2g}$ manifolds in any frame but with splittings induced by the rotation,
which generates non-octahedral but still only $l=4$ field components. 
\item
%(ii) 
The rigid rotation and tilting of the octahedra keeps the octahedra symmetric but deforms the unit cell and AB ionic cages with it.
This not only changes the $l=4$ components but also induces 
$l=2$ field components
that can couple to the JT ion much more strongly (see discussion on $D_2$ vs $D_4$ below \eref{eq:JTeg} in Sec.~\ref{sec:JTtertagonal}).
So, the magnitude of the net effect of the octahedral rotations is not limited by the relative sizes of the octahedral fields of AB and O in the cubic case (see Sec.~\ref{sec:relativeVoctABO}).
\end{enumerate}




\com{
There are two different ways the crystal field $V(\vec r)$
 is deformed from the ideal case, $V_{voct}(\vec r)$.
\begin{enumerate}
\item The relative rotations of the AB and O frames due to
the rigid rotation and tilting of the oxygen octahedra removes the
choice of a set of d-orbitals where the effect of 
AB and O cages is the same. 
We no longer just get eg-t2g in any frame but with splittings induced by the rotation.
This can generate non-octahedral but still $l=4$ field components. 
\item
The rigid rotation and tilting of the octahedra keeps the octahedra symmetric but deforms the unit cell and AB ionic cages with it.
This not only changes the $l=4$ components but also induces 
$l=2$ field components
that can couple to the JT ion much more strongly (see discussion on $D_2$ vs $D_4$ below).
So, the magnitude of the net effect of the octahedral rotations is not limited by the relative sizes of the octahedral fields of AB and O in the cubic case (see Sec.~\ref{sec:relativeVoctABO}).
\end{enumerate}

}

Since $V_{O}(\vec r)$ is relatively stronger, 
we can take the octahedra's local frames (that are rotated with them)
to first describe the effect of $V_{O}(\vec r)$, i.e.,
splitting of degenerate d-orbitals into $e_g, t_{2g}$ manifolds with 
a gap $\Delta_o$,
and then
consider the effect of $V_{AB}(\vec r)$ on these manifolds.
$V_{AB}(\vec r)$ will couple and mix the orbitals in the two manifolds.
Since %$\Delta_AB << \Delta_O$,
$f_{q_A} \ll 1$,
we can even ignore the coupling between the states belonging to
different manifolds that are $\Delta_o$ apart.
This also makes it possible to analytically solve the problem in two limiting cases discussed later.


\com{
The total crystal field potential at the central B atom that we consider
 is given by
\begin{align}
V(\vec r) &= V_O(\vec r) + V_{AB}(\vec r),
\end{align}
where
$V_{AB}(\vec r)=V_{A}(\vec r)+V_{B}(\vec r)$
is due to A and B ion cages.
\az{repetition.... cubic case has similar expressions.... }
}

It is worth reminding that
$V(\vec r) $ a scalar quantity in the position space but a vector 
%$V$ is as a vector %of length $(l_{max}+1)^2=25$ for $l_{max}=4$.
in the spherical harmonics bases, so
we can calculate it in one bases as per our convenience and then transform it to another.
%Its components in the new basis.
\begin{align}
V(\vec r) &= \sum_{l,m} V_{lm} r^l Y_{lm}(\hat r)
= \sum_{l,m} \tilde V_{lm} r^l \tilde Y_{lm}(\hat r),\\
%\tilde V_{lm} &= \sum_{l',m'}\mathcal{R}_{lm,l'm'}V_{l'm'},
\tilde V_{lm} &= \sum_{m'=-l}^{l}\mathcal{R}^{l}_{m,m'}V_{lm'},
\end{align}
where 
$\mathcal{R}^{l}=\mathcal{R}^{l}(\alpha,\beta)$ is the rotation matrix for real spherical harmonics 
of degree $l$. % (see appendix~\ref{sec:appendix}).
It can be calculated using the method 
devised by Ivanic and Ruedenberg~\cite{IvanicJPC96}.
The multipole components of $V_O(\vec r)$ in the rotated frame do not change with the rigid octahedral rotations because, by definition, the octahedron itself rotates with its frame.
These can be directly read off \eref{eq:voct}.
Similarly, it is easier to obtain the components of $V_{AB}(\vec r)$
in the unrotated frame so we do that first and later transforms them to obtain the rotated frame components.
That is,
\begin{gather}
\tilde V_{lm} = \tilde V_{lm}^O + \sum_{m'=-l}^{l}\mathcal{R}^{l}_{m m'} V_{lm'}^{AB},\\
V_{lm}^{AB} = \left(\frac{q_A}{a} \right)V_{lm}^{A}   
+ \left(\frac{q_B}{a}\right)V_{lm}^{B},
\end{gather}
%where $q_A + q_B = 3q_o$ ensures charge neutrality.
where the quantities bearing a tilde on top belong to the rotated (octahedron's) frame.
As described above, 
$\tilde V_{lm}^O$ are simply octahedral field components 
with $l=4,m=0,4$ (see \eref{eq:voct}).
However, $V_{lm}^{AB}$ contain the effect of the deformation of the crystal structure 
%on AB ion cages due to the octahedral rotations
so it now has not only other $l=4$ components but also $l=2$ components; see below.
A procedure to obtain these multipole components, without
actually calculating the projection of the potential $V_{AB}(\vec r)$ 
onto the spherical harmonics, is described in the appendix. %~\ref{append:multipoles}.



We find that the splittings due to octahedral rotation and tilting are practically the same for all B ions
(but their eigenstates differ considerably, which creates interesting orbital ordering as discussed later in the discussion in Sec.~\ref{sec:discussion}),
so we consider the octahedron $1$ with central B ion at $\textbf{a}/2$
in the calculations that follow.
We obtain the following exact expressions for $V_{lm}^{A}$,
\begin{widetext}
%\rev{$V(r)=q\sum_{i}(\sqrt{ ..... r\cos... })^{-1}$}
%\az{full potential of the charges in deformed cage: manually refined}:
\begin{align}
V^A_{2,-2}&= 64 \sqrt{\frac{3 \pi }{5}} \cos ^2\beta  \left(\frac{1}{\left(\cos ^2\alpha +2 \cos ^2\beta \right)^{5/2}}-\frac{\sec ^3\alpha }{(\cos 2 \beta +2)^{5/2}}\right),\\
V^A_{2,0}&= 32 \sqrt{\frac{\pi }{5}} \left(\frac{2 \sec ^3\alpha  \sin ^2\beta }{(\cos 2 \beta +2)^{5/2}}-\frac{32 \sqrt{\frac{\pi }{5}} (\cos 2 \beta -\cos 2 \alpha )}{\left(\cos ^2\alpha +2 \cos ^2\beta \right)^{5/2}}\right),\\
V^A_{4,-2}&= \frac{128\sqrt{5 \pi } }{3} \cos ^2\beta  \left(\frac{3 \cos 2 \alpha -\cos 2 \beta +2}{\left(\cos ^2\alpha +2 \cos ^2\beta \right)^{9/2}}+\frac{\sec ^5\alpha  (\cos 2 \beta -5)}{(\cos 2 \beta +2)^{9/2}}\right),\\ \nonumber
V^A_{4,0}&= \frac{16\sqrt{\pi } }{3} \left(
\frac{\sec ^5\alpha  (36 \cos 2 \beta -3 \cos 4 \beta +23)}{(\cos 2 \beta +2)^{9/2}}
-
\frac{8 \cos 2 \alpha  (3 \cos 2 \beta +2)-2 \cos 4 \alpha+12 \cos 2 \beta -3 \cos 4 \beta +9}{\left(\cos ^2\alpha +2 \cos ^2\beta \right)^{9/2}}
\right),\\
V^A_{4,4}&=-\frac{128\sqrt{35 \pi } }{3} \cos ^4\beta  \left(\frac{1}{\left(\cos ^2\alpha +2 \cos ^2\beta \right)^{9/2}}+\frac{\sec ^5\alpha }{(\cos 2 \beta +2)^{9/2}}\right),
\end{align}
\end{widetext}
whereas $V_{lm}^{B}$ are given by,
%\az{full potential of the charges in deformed cage: }:
\begin{align}
V^{B}_{2,-2}&= 64 \sqrt{\frac{3 \pi }{5}}  \frac{\sin ^2\alpha  \sec ^3\beta }{(\cos 2 \alpha +3)^{5/2}},
 \\ V^{B}_{2,0}&=  4 \sqrt{\frac{\pi }{5}} \left(\sec ^3\alpha -\frac{8 \sec ^3\beta }{(\cos 2 \alpha +3)^{3/2}}\right),\\
 V^{B}_{4,-2}&=  - \frac{128 \sqrt{5 \pi } }{3} \frac{\sin ^2\alpha  \sec ^5\beta }{ (\cos 2 \alpha +3)^{7/2}},\\ 
V^{B}_{4,0}&= \frac{4\sqrt{\pi } }{3} \left(\frac{24 \sec ^5\beta }{(\cos 2 \alpha +3)^{5/2}}+\sec ^5\alpha \right),\\
 V^{B}_{4,4}&=  \frac{16 \sqrt{35 \pi } }{3} \frac{(20 \cos 2 \alpha -\cos 4 \alpha +13) 
 \sec ^5\beta }
 {(\cos 2 \alpha +3)^{9/2}}.
 \end{align}


% expansion of V^{AB} at small alpha,beta is not useful because
% we still have to Rotate them that brings in more alpa beta dependance
\com{
%A
\begin{align}
\text{Vlm}(2,-2)&=  \frac{64}{81} \sqrt{\frac{\pi }{5}} \alpha ^2 \left(\beta ^2-6\right),\\
\text{Vlm}(2,0)&=  \frac{64}{27} \sqrt{\frac{\pi }{15}} \left(2 \left(\alpha ^2+3\right) \beta ^2-3 \alpha ^2\right),\\
\text{Vlm}(4,-2)&=  \frac{1}{243} (-256) \sqrt{\frac{5 \pi }{3}} \alpha ^2 \left(9 \beta ^2+5\right),\\\text{Vlm}(4,0)&=  \frac{1}{243} (-128) \sqrt{\frac{\pi }{3}} \left(5 \alpha ^2 \left(11 \beta ^2+4\right)+30 \beta ^2+14\right),\\
\text{Vlm}(4,4)&=  \frac{1}{243} (-128) \sqrt{\frac{35 \pi }{3}} \left(\alpha ^2 \left(5 \beta ^2+4\right)+2 \left(\beta ^2+1\right)\right)
 \end{align}
% B
\begin{align}
\text{Vlm}(2,-2)&=  \sqrt{\frac{3 \pi }{5}} \alpha ^2 \left(3 \beta ^2+2\right),\\
\text{Vlm}(2,0)&=  \frac{1}{2} (-3) \sqrt{\frac{\pi }{5}} \left(\alpha ^2 \left(3 \beta ^2-2\right)+4 \beta ^2\right),\\
\text{Vlm}(4,-2)&=  -\frac{1}{6} \sqrt{5 \pi } \alpha ^2 \left(5 \beta ^2+2\right),\\
\text{Vlm}(4,0)&=  \frac{1}{24} \sqrt{\pi } \left(5 \alpha ^2 \left(15 \beta ^2+22\right)+60 \beta ^2+56\right),\\
\text{Vlm}(4,4)&=  \frac{1}{24} \sqrt{35 \pi } \left(5 \alpha ^2+4\right) \left(5 \beta ^2+2\right)
 \end{align}
}



%\subsection{Crystal Field Hamiltonian}

\com{ Considering an octahedron,.... which one?
 why only one but not all four?
 spectra of all octahedra are similar...???
}

\com{V20, JT:\\
V2-2, V4-2, couple yz and zx.
But... rotation.... produce all m components in general, ....
}


The Hamiltonian of the central ion (JT ion) including 
the crystal field of the AB cages 
in the octahedron's frame %(rotated frame)
are given by,
\begin{align}
\tilde H_{m' m''} &= \sum_{l,m} \tilde V_{l m} \tilde C_{l}^{m'm''m} D_{l} ,
%C_{l}^{m'm''m} &= \braket{\tilde Y_{2,m'}|\tilde Y_{l,m}|\tilde Y_{2,m''}},
\end{align}
where %$C_{l}^{m'm''m}$ 
$\tilde C_{l}^{m'm''m}=\braket{\tilde Y_{2m'}|\tilde Y_{lm}|\tilde Y_{2m''}}= C_{l}^{m'm''m}$. 
% are again the Gaunt coefficients for the real spherical harmonics.





\begin{figure}[htbp]
\begin{center}
%\includegraphics[width=1.0\linewidth]{/Users/maz/repos/perovskite/calc/analytical-structure-param/fig-data/energy-vs-beta-alpha-0-width-row}
\includegraphics[width=1.0\linewidth]{energy-vs-beta-alpha-0-width-row}
\caption{
Splitting in $t_{2g}$ manifold induced by the
octahedral rotations.
%The changes in the energies of $t_{2g}$ states 
$E_{xy},E_{yz},E_{zx}$ 
%and the average of $t_{2g}$ energy
and $E_0^{t_{2g}}$
as a function of the rotation angel $\beta$
at $\alpha=0$ (no tilting),
$\Delta_o=4q_o/15a$ ($D_4=1/200$), $D_2=\sqrt{D_4}$,
and three values of $q_A/q_o$.
(a)
The B cage dominates at $q_A/q_o=1/2$;
it lowers $xy$ state and raises $yz,zx$ states
with increasing $\beta$.
As $q_A/q_o$ increases to $1$, 
(b)
A cage almost cancels the effect of B cage reducing the splitting,
while it dominates at $q_A/q_o=2$ (c) where it switches the order of the states. 
}
\label{fig:t2g-alpha0}
\end{center}
\end{figure}



\section{Results: Single particle spectrum and JT instability}
\label{sec:results}
 
 

Before we present our results,
let us briefly describe 
the kind of behaviour that would indicate a JT instab against the tilting and rotations of the octahedra,
and how it relates to the standard JT problem.
As describe in Sec.~\ref{sec:JT},
the optimum structure can have a finite distortion
as long as 
the splitting in the single particle spectrum has a different dependence on the distortion
than the elastic energy associated with the changes in the unit cell volume.
\rev{Here, the unit cell volume
 depends on the ``distortions'' $\alpha,\beta$ 
as ${v=4a^3 \cos ^2\alpha \cos ^2\beta}$,
and, at $\alpha,\beta\ll1$,
it changes as $\sim a^3\left(\alpha ^2+\beta ^2\right)$.
The elastic energy in the harmonic approximation thus goes
as $\sim\left(\alpha ^2+\beta ^2\right)^2$,
which is \emph{quartic} in $\alpha,\beta$.
So, 
in contrast to the case of
elongations or 
compressions  
of the octahedra~\cite{KhomskiiCR21},
the behavior we are looking for in the electronic spectrum 
is not linear
but quadratic, 
which would compete with such a quartic elastic energy
 to produce a finite distortion in the optimum structure.
Thinking of squared angles 
$\alpha ^2,\beta ^2$
as the basic distortion variables,
the order of the electronic and elastic energy terms
matches to the standard JT problem.
}



\com{In contrast to 
%hypothetical 
elongations or 
compressions---$a\delta_1,a\delta_2$ 
of an octahedron along two of its axes---where the 
 the unit cell volume changes
as $\sim a^3\left(\delta_1+\delta_2\right)$
and thus
the elastic energy in the harmonic approximation goes
as $\sim\left(\delta_1+\delta_2\right)^2$,
the unit cell volume
in case of octahedral rotation and tilting
 depends on the ``distortions'' $\alpha,\beta$ 
as ${v=4a^3 \cos ^2\alpha \cos ^2\beta}$,
and, at $\alpha,\beta\ll1$,
it changes as $\sim a^3\left(\alpha ^2+\beta ^2\right)$
leading to a \emph{quartic} elastic energy that should roughly behave as
$\sim\left(\alpha ^2+\beta ^2\right)^2$.
So, the behavior we are looking for in the electronic spectrum 
is not linear
but quadratic, 
which would compete with such a quartic elastic energy
 to produce a finite distortion in the optimum structure.
Thinking of squared angles 
$\alpha ^2,\beta ^2$
as the basic distortion variables,
the order of the electronic and elastic energy terms
matches to the standard JT problem.
}









 
\com{
Before we present our results,
let's briefly describe 
the kind of behaviour that would indicate a JT instability 
against the tilting and rotations of the octahedra,
and how it relates to the standard JT problem.
As describe in Sec.~\ref{sec:JT},
the optimum structure can have a finite distortion
as long as 
the splitting in the single particle spectrum has a different dependence on the distortion
than the elastic energy associated with the changes in the unit cell volume.

%\subsection{Dependence of electronic spectrum on the ``distortion'' angles}
In case of \az{names? Tetragonal distortion, or quadrupolar distortion, }
elongation or compression of a cube of size $a$ along two of its edges, say, by $a\delta_1,a\delta_2$,
the change in the unit cell volume would be linear in $\delta_{1,2}$
as $\sim a^3\left(\delta_1+\delta_1\right)$.
So the elastic energy %(for an isotropic system) 
changes as $\sim\left(\delta_1+\delta_1\right)^2$.
A linear change ($\propto \delta_{1/2}$) in the electronic spectrum is required to produce
 a finite distortion in the optimum structure (or the ground state of full electron-lattice/nuclear problem).
For example, in the case of tetragonal distortion of an octahedron,
 the total energy has a form
$E=-\rho\delta + K \delta^2$, where $\rho$ and $K$ are constants 
describing the changes in the electronic and elastic energies,
and would be optimum at $\delta=\rho/2K$.
In our case, however,
the unit cell volume depends on the ``distortions'' $\alpha,\beta$ as ${v=4a^3 \cos ^2\alpha \cos ^2\beta}$,
and, at $\alpha,\beta\ll1$,
it changes as $-4a^3\left(\alpha ^2+\beta ^2\right)$.
So, the behaviour we are looking for in the electronic spectrum does not have to be linear 
but quadratic, which would compete with a quartic elastic energy $\left[ \sim\left(\alpha ^2+\beta ^2\right)^2 \right]$
 to produce a finite distortion in the optimum structure.
}


%\az{Focus on t2g first:}\\

We will first focus on the $t_{2g}$ manifold
and see how the potential of the AB cages split it when 
the octahedra tilt and rotate.

\subsection{$t_{2g}$ manifold} %states and energies}
\label{sec:t2g}

Let's consider three cases separately, only rotation, only tilting, and both rotation and tilting.

\subsubsection{Only rotation: $\alpha=0,\beta\neq 0$}
\label{sec:alpha0}

%We find that $xy$ is coupled to $x^2-y^2$
%$t_{2g}$ and $e_{g}$ manifolds are coupled.

Ignoring the coupling between the $e_{g}$ and $t_{2g}$ manifolds (between $xy$ and $x^2-y^2$ at $\alpha=0$),
their Hamiltonians turn out to be diagonal.
We can write the single particle energies of $t_{2g}$ states
as
\begin{align}
\label{eq:theta0-Es}
(E_{xy},&E_{yz},E_{zx})= (2, -1, -1)
\left[\frac{}{} \Delta_{2}  D_2 + \Delta_4 D_4\right],
 \end{align}
 where $E_{yz}$ and $E_{zx}$ are degenerate and 
\begin{align}
\Delta_{l} =&  \left(q_A \Delta_{l}^A + q_B\Delta_{l}^B\right)/a,~(l=2,4),\\
\Delta_{2}^A =& 
%from A
\frac{64 \sin ^2\beta }{7 (\cos 2 \beta +2)^{5/2}} % simplifies.... from:
 %\frac{16 (3 \sin \beta+\sin 3\beta)^2}{7 (\cos 2\beta+2)^{9/2}},\\
 \simeq \frac{64}{63 \sqrt{3}}\beta ^2,\\
 \nonumber
\Delta_4^A =&
 \frac{160 \sin ^2\beta}{63 (\cos 2\beta+2)^{9/2}} \times \\
&  (125 \cos 2\beta+42 \cos 4\beta+7 \cos 6\beta+82)\\
\simeq& \frac{40960}{5103 \sqrt{3}}\beta ^2,\\
% \Delta_4^A =&
% \frac{160 \sin ^2\beta (125 \cos 2\beta+42 \cos 4\beta+7 \cos 6\beta+82)}{63 (\cos 2\beta+2)^{9/2}},\\
 %from B
 \label{eq:theta0-delB2}
\Delta_{2}^B =& \frac{2}{7} \left(1-\sec ^3\beta\right)
\simeq -\frac{3}{7}\beta ^2,\\
\label{eq:theta0-delB4}
  \Delta_4^B =&\frac{5}{126} \left((7 \cos 4\beta-3) \sec ^5\beta-4\right)
  \simeq -\frac{115}{63}\beta ^2.
 %\left( \Delta_{2}  D_2 + \Delta_4 D_4\right),\\
 \end{align}
 We see that $\Delta_{2/4}^A>0$ while $\Delta_{2/4}^B<0$,
 revealing a competition between the two terms 
 that would determine the sign and size of the term
  in the square bracket in \eref{eq:theta0-Es} 
%  (hence the ground state and splitting) 
  as $q_A$ (and hence $q_B$)
  is varied.
The average $t_{2g}$ energy
 also shifts from
$-2/5\Delta$ by an amount
%\rev{E0t2g or change in it? how to define the expressions... plotted is always change in it...}
\begin{align}
\label{eq:theta0-E0}
E_{0}^{t_{2g}} = \left(q_A E_{0}^A + q_BE_{0}^B\right)D_4/a
-(2/5)f_{q_A}\Delta_{o},
 \end{align}
 where
\begin{align}
\nonumber
E_{0}^A =&
 \frac{8 (92 \cos 2\beta+24 \cos 4\beta+20 \cos 6\beta+5 \cos 8\beta+51)}{9 (\cos 2\beta+2)^{9/2}}\\
 \simeq& \frac{512}{243 \sqrt{3}}-\frac{2560 \beta ^2}{729 \sqrt{3}},\\
 \nonumber
E_{0}^B=&
 -\frac{1}{18} \left((5 \cos 4\beta+3) \sec ^5\beta+4\right)\\
 \simeq& -\frac{2}{3}+\frac{10 \beta ^2}{9}.
 \end{align}
 
The results in Eqs.~\ref{eq:theta0-Es},\ref{eq:theta0-E0}
are summarised in Fig.~\ref{fig:t2g-alpha0},
where $E_{xy}$, $E_{yz/zx}$ and $E_0^{t_{2g}}$ are shown
as a function of $\beta$
at three values of $q_A/q_o=1/2,1,2$.
We have 
assumed $D_2=\sqrt{D_4}$ and
taken $\Delta_o=4q_o/15a$ throughout this paper,
which fixes the value of $D_4=1/200$, 
and corresponds to $\Delta_o= 2.073$ eV at 
$q_o=2$ e and $a=7.0$ bohr.
We present the energies as a percentage of $\Delta_o$
by calculating all quantities at the above values of $q_o,a$.
Fig.~\ref{fig:t2g-alpha0} shows that
the competition between A and B cages
determines the ground state and the size of the splitting between the 
$t_{2g}$ states
$E_{xy}$ and $E_{yz/zx}$.
In Fig.~\ref{fig:t2g-alpha0}(a),
where $q_A/q_o=1/2$ ($q_B/q_o=5/2$),
the effect of B cage dominates.
Since it's negative (see Eqs.~\ref{eq:theta0-delB2},\ref{eq:theta0-delB4}),
it reverses the order in \eref{eq:theta0-Es}
and $E_{xy}$ becomes lower than $E_{yz/zx}$, by about $6\%$ of $\Delta_o$ at $\beta=15^\circ$.
When A cage dominates at large $q_A/q_o$,
as shown in Fig.~\ref{fig:t2g-alpha0}(c) for $q_A/q_o=2$,
$E_{yz/zx}$ are the single particle ground states
 (with a splitting of the same order), %$\sim6\%$ of $\Delta_o$ at $15^\circ$),
whereas the net effect of both cages
is reduced in Fig.~\ref{fig:t2g-alpha0}(b)
that corresponds to $q_A/q_o=1$
and we only obtain a splitting of $\sim2\%$ of $\Delta_o$ at $\beta=15^\circ$.
%between $E_{xy}$ and $E_{yz/zx}$ levels.
Obviously, there will also be a point near 
$q_A/q_o=1$ where A and B completely 
cancel the effect of each other (not shown in Fig.~\ref{fig:t2g-alpha0}).
We also see that $E_0^{t_{2g}}$ changes from a small negative to a small positive value
with an increase in $q_A/q_o$ as we
go from Fig.~\ref{fig:t2g-alpha0}(a)
to Fig.~\ref{fig:t2g-alpha0}(c).
This is because the rotation suppresses the
effect of AB cages on the
gap $\Delta$ between $e_{g}$ and $t_{2g}$.
Since A supports O while B opposes it,
$\Delta$ slightly widens at small $q_A$ and shrinks at large $q_A$,
shifting the $t_{2g}$ manifold slightly down or up with $\beta$.

These results also indicate that in case of a superlattice with $q_A$ alternating
between the two regimes shown in Figs.~\ref{fig:t2g-alpha0}(a,c),
we should obtain an alternating orbital order if the octahedra are rotated.





\com{
$\eta=\left(\frac{a\Delta_{O}}{q_o}\right)=\frac{4}{15}$
at $D_4=1/200$\\
$a=7$ bohr, and $\braket{r^4}=11.5809$ gives $2eV$ crystal field gap: $D_4=11.5809/a^4=0.004823365264473\simeq0.005=1/200=5\times10^{-3}$,
$D_2=\sqrt{D4}=0.0694503$,
$D_2/D_4\simeq 14$.
}
\com{$D_2/D_4$ ranges between $15$-$20$ for typical gaps of $2$-$1$ eV and $a\simeq 7$ bohr (and $q_o=2e$). 
It decreases with increasing gap as $1/\sqrt(gap)$, but is still $10$ at $gap=3.5$eV.
}






\subsubsection{Only tilting: $\alpha\neq 0,\beta = 0$}
\label{sec:t2g-beta0}

In this case, the situation gets slightly complicated as
the couplings between the three states become finite. 
However, 
the Hamiltonian still has a simple structure, 
\begin{align} 
\tilde H = 
\left(
\begin{array}{ccc}
 E_{xy} & g & -g \\
 g & E_{yz} & \lambda  \\
 -g & \lambda  & E_{yz} \\
\end{array}
\right),
\end{align}
where the matrix elements have relatively long expressions and 
only an approximation at small $\alpha,\beta$ is given later in this section.
%\com{are not given here \az{[?? can be found in the appendix.]} An approximation at small $\alpha,\beta$ is given later in this section.}
$\tilde H$ can be simplified by rotating the bases in $\{\ket{yz},\ket{zx}\}$ space
to obtain bright and dark states
as follows.
\begin{align}
\ket{B} &= \left(\ket{yz} - \ket{zx}\right)/\sqrt{2},\\
\ket{D} &=  (\ket{yz} + \ket{zx})/\sqrt{2}.
\end{align}
The state $\ket{D}$ at energy ${E_D= E_{yz}+\lambda}$ decouples 
from the rest (so is already an eigenstate of $\tilde H$) 
and the state $\ket{B}$ at energy ${E_{yz}-\lambda}$ obtains the enhanced collective coupling $\sqrt{2} g$. 
\com{
\begin{align}
\tilde H_{Bright} = 
\left(
\begin{array}{cc}
E_{xy} & \sqrt{2} g \\
\sqrt{2} g & E_{yz}-\lambda
\end{array}
\right).
\end{align}
The Hamiltonian can be solved analytically.
}
The Hamiltonian can thus be solved analytically to
obtain the remaining two
eigenstates
\begin{gather}
%\ket{+} = \cos\gamma \ket{xy} + \sin\gamma \left(\ket{yz} - \ket{zx}\right)/\sqrt{2},\\
%\ket{-} = \sin\gamma \ket{xy} - \cos\gamma \left(\ket{yz} - \ket{zx}\right)/\sqrt{2},\\
\ket{+} = \cos\gamma \ket{xy} + \sin\gamma \ket{B},\\
\ket{-} = \sin\gamma \ket{xy} - \cos\gamma \ket{B},\\
\tan2\gamma = \sqrt{8}g/(E_{xy}-E_{yz}+\lambda),
\end{gather}
at 
energies
\begin{gather}
E_{\pm} = \frac{E_{xy}+E_{yz}-\lambda}{2}  \pm \frac{1}{2}\sqrt{(E_{xy}-E_{yz}+\lambda)^2 + 8g^2 }.
%\delta = E_{xy}-e_z+\lambda.
\end{gather}






Since we are only interested in small tilt and rotation,
an expansion of the matrix elements of the Hamiltonian about 
%${\alpha=0=\beta}$
${\alpha,\beta=0}$
%${(\alpha,\beta)=(0,0)}$
to the leading order turns out to be quite good an approximation.
The matrix elements in this case simplify a lot.
Defining a vector ${Q = \left( q_A D_2, q_A D_4, q_B D_2,q_B D_4  \right)}$,
 we can write them as
\begin{align}
E_{xy}=& Q.
\left(-\frac{64 \alpha ^2}{63 \sqrt{3}},-\frac{512 \left(15 \alpha ^2+7\right)}{1701 \sqrt{3}},\frac{3\alpha ^2}{7} ,\frac{5 \alpha ^2}{7}+\frac{2}{3}\right),\\
E_{yz}=& Q.
\left(\frac{32 \alpha ^2}{63 \sqrt{3}},\frac{512 \left(25 \alpha ^2-7\right)}{1701 \sqrt{3}},-\frac{3 \alpha ^2}{14},-\frac{85 \alpha ^2}{42}+\frac{2}{3}\right),\\
g=& Q.
\left(-\frac{80}{63} \sqrt{\frac{2}{3}} \alpha ^3,0,\frac{15 \alpha ^3}{14 \sqrt{2}},\frac{5 \alpha ^3}{2 \sqrt{2}}\right),\\
\lambda=& Q.
\left(\frac{64 \alpha ^2}{63 \sqrt{3}},-\frac{2560 \alpha ^2}{1701 \sqrt{3}},-\frac{3 \alpha ^2}{7},\frac{40 \alpha ^2}{21}\right).
%E_0=& Q.\left(-\frac{16 \alpha ^2}{21 \sqrt{3}},\frac{256 \left(15 \alpha ^2-14\right)}{1701 \sqrt{3}},\frac{9 \alpha ^2}{28},\frac{2}{3}-\frac{45 \alpha ^2}{28}\right),
%Q =& \left( q_A D_2, q_A D_4, q_B D_2,q_B D_4     \right).
%E_0(0,0)=& Q.
%\frac{1}{2} \left(\frac{4 \text{D4} \text{qB}}{3}-\frac{1024 \text{D4} \text{qA}}{243 \sqrt{3}}\right)
\end{align}





\begin{figure}[htbp]
\begin{center}
%\includegraphics[width=1.0\linewidth]{/Users/maz/repos/perovskite/calc/analytical-structure-param/fig-data/energy-vs-alpha-beta-0-width-row-character}
\includegraphics[width=1.0\linewidth]{energy-vs-alpha-beta-0-width-row-character}
\caption{
Splitting and mixing in $t_{2g}$ manifold induced by octahedral tilting.
(a-c) 
$E_D, E_{\pm}$ and $E_0^{t_{2g}}$
as a function of $\alpha$
at 
$\beta=0$;
other parameters are the same as in Fig.~\ref{fig:t2g-alpha0}.
(d-f) Weight $W_{xy}$ of $xy$ orbital in 
$\ket{E_{\pm}}$ states 
shown in (a-c).
The $xy$ orbital predominantly
contributes to 
$\ket{E_{+}}$
at $q_A/q_o=1/2,1$ (a,b,d,e), 
but to   
$\ket{E_{-}}$
at $q_A/q_o=2$ (c,f).
}
\label{fig:t2g-beta0}
\end{center}
\end{figure}



$E_D, E_{\pm}$ relative to their average
and the weight of $xy$ state $W_{xy}$ in $E_{\pm}$,
as a function of $\alpha$ at $q_A/q_o=1/2,1,2$ (and $\Delta_o=4q_o/15a$)
are shown in Fig.~\ref{fig:t2g-beta0}.
The average shifts slightly with $\alpha,q_A$ by 
$E_0^{t_{2g}}= \left(E_{xy} + 2 E_{yz}\right)/3-(2/5)f_{q_A}\Delta_{o}$,
again reflecting 
a change in the gap $\Delta$,
and is also included in Fig.~\ref{fig:t2g-beta0}.
Apart from the mixing between the three states $xy,yz,zx$ in $\ket{\pm}$ at finite $\alpha$, 
Fig.~\ref{fig:t2g-beta0} shows a similar trend as in Fig.~\ref{fig:t2g-alpha0}
but with an interesting twist.
%Fig.~\ref{fig:evals-alpha} reveals another interesting point.
In contrast to $\alpha=0$ case discussed in the previous section,
$W_{xy}$ is the largest in the highest (lowest) energy state at small (large) $q_A/q_o$,
as shown in Fig.~\ref{fig:t2g-beta0}(a,c).
$\alpha$ and $\beta$ would thus tend to produce different ordering
(details in the next section).
The splittings are also about the same magnitude, between $4$-$6\%$ at $\alpha=15^\circ$
at $q_A/q_o=1/2,2$ [Fig.~\ref{fig:t2g-beta0}(a,c)] and about $2\%$ at $q_A/q_o=1$ [Fig.~\ref{fig:t2g-beta0}(b)].
The mixing is sizeable but still far from perfect ($50/50\%$) even at the largest $\alpha$ shown ($W_{xy}=0.25$-$0.35$ at $\alpha=20^\circ$).
% At unrealistically large $\alpha\sim30-40^\circ$, 
Overall, at finite $\alpha$,
the lowest energy state is either purely $xy/yz$,
or contains at least some fraction of these,
but it is never pure $xy$.






\begin{figure}[htbp]
\begin{center}
%\includegraphics[width=1.0\linewidth]{/Users/maz/repos/perovskite/calc/analytical-structure-param/fig-data/energy-vs-beta-alpha-10-width-row-beta-0-15}
\includegraphics[width=1.0\linewidth]{energy-vs-beta-alpha-10-width-row-beta-0-15}
\caption{
Splitting in $t_{2g}$ manifold due to octahedral rotation and tilting.
$t_{2g}$ energies as a function of $\beta$ at $\alpha=10^\circ$
at the same values of other parameters as in Figs.~\ref{fig:t2g-alpha0},\ref{fig:t2g-beta0}.
There are two anticrossings between the three states around $\beta\sim\alpha$.
The character of the states depends on 
the competition between the two angles and 
the two ion cages.
}
\label{fig:t2g-ge-Es}
\end{center}
\end{figure}


\begin{figure}[htbp]
\begin{center}
%\includegraphics[width=1.0\linewidth]{/Users/maz/repos/perovskite/calc/analytical-structure-param/fig-data/wts-energy-vs-beta-alpha-10-width-row-beta-0-15}
\includegraphics[width=1.0\linewidth]{wts-energy-vs-beta-alpha-10-width-row-beta-0-15}
\caption{
The dependence of the character of the $t_{2g}$ states
on the competitions between
$\alpha,\beta$ and 
A,B ion cages.
Weights of the three $t_{2g}$ orbitals in the three 
$t_{2g}$ eigenstates 
shown in
Fig.~\ref{fig:t2g-ge-Es}.
%(at the same parameters).
At $q_A=q_o/2$, B cage dominates and
$\ket{E_1}$ is predominantly $yz,zx$ at $\beta\sim 0\ll\alpha=10^\circ$, but 
becomes $\sim xy$ at $\beta\gtrsim10^\circ$ (a).
The situation flips when A cage dominates at $q_A=2q_o$ (d).
Similarly, the character of $\ket{E_3}$ at 
$q_A=q_o/2$
resembles 
that of $\ket{E_1}$ at $q_A=2q_o$  [compare (c) with (d)], and vice versa [compare (f) with (a)].
The character of the middle state $\ket{E_2}$ (b,e)
becomes $\sim xy$ only around the ``resonance'' region
$\beta\sim \alpha$.
}
\label{fig:t2g-ge-wts}
\end{center}
\end{figure}


\subsubsection{Rotation and tilting: $\alpha,\beta\neq 0$}
\label{sec:t2g-gen}

The $t_{2g}$ Hamiltonian in the general case $\alpha,\beta\neq 0$ cannot be solved analytically
 so there is no longer any advantage of ignoring the coupling between 
 the $e_{g}$ and $t_{2g}$ manifolds.
Even though their effect should be negligible due to a large gap $\Delta$ between the two manifolds,
here we keep these couplings 
and numerically solve
the full Hamiltonian.
Figure~\ref{fig:t2g-ge-Es}
shows the energies at $\alpha=10^\circ$
as a function of $\beta$
at $q_A/q_o=1/2,2$ and $\Delta_o=4q_o/15a$ again.
All states in the $t_{2g}$ mix up to make the eigenstates $\{\ket{E_i}\}$
at energies $\{E_i\}, i=1,2,3$.
%\az{E0, can we show E0 only in the first fig and not all of them?}
At $\beta=0$, these results correspond to $\alpha=10^\circ$ 
in Figs.~\ref{fig:t2g-beta0}(a,c),
so the lowest state at $\beta=0$
here is $\ket{D}$ in Fig.~\ref{fig:t2g-ge-Es}(a)
while it is the highest in Fig.~\ref{fig:t2g-ge-Es}(b).
We see two anticrossings or level repulsion in either of Figs.~\ref{fig:t2g-ge-Es}(a,b)
in different orders.
In Fig.~\ref{fig:t2g-ge-Es}(a), 
the anticorssing between the highest two states occurs at smaller $\beta$,
while the situation is the opposite in Fig.~\ref{fig:t2g-ge-Es}(b) where 
it occurs between the lowest pair at smaller $\beta$.
From the results in the previous two sections,
we know that not only A and B but also $\alpha$ and $\beta$ tend to establish different orders.
To get more insight, the weights of these three $t_{2g}$ states in the three eigenstates 
% in Fig.~\ref{fig:t2g-ge-Es}
are shown in Fig.~\ref{fig:t2g-ge-wts}.
Let us first focus on the left column ($q_A/q_o=1/2$). 
Fig.~\ref{fig:t2g-ge-wts}(a)
shows that, while $\ket{E_1}\to \ket{D}$ as $\beta\to 0$,
$\ket{E_1}$ practically stays in $yz,xz$ subspace
for $\beta<7^\circ$ and it quickly becomes $xy$
as $\beta\to\alpha=10^\circ$.
Figs.~\ref{fig:t2g-ge-wts}(b,c)
show a complimentary behaviour 
between $xy$ and $yz,xz$,
where $\ket{E_3}$
transforms from almost pure $xy$ at $\beta=0$
to almost complete $yz,xz$ at $\beta\gtrsim\alpha=10^\circ$.
The right column
in Fig.~\ref{fig:t2g-ge-wts} ($q_A/q_o=2$),
shows the opposite behaviour 
about the ``resonance'' $\beta\sim\alpha$.
Figs.~\ref{fig:t2g-ge-wts}(b,e)
show an interesting aspect,
the $W_{xy}$ peaking in the middle state $\ket{E_2}$
before it is transferred between $\ket{E_1}$ and $\ket{E_2}$
around $\beta\sim\alpha$,
i.e., at $\beta$
between the two anticrossings in 
Fig.~\ref{fig:t2g-ge-Es},
which makes sense if we imagine the continuation of various energy surfaces/curves
in 
the spectra in Fig.~\ref{fig:t2g-ge-Es}
ignoring their anticorssings:
$xy$ starts from the top and goes down in 
Fig.~\ref{fig:t2g-ge-Es}(a),
while it starts from the bottom and goes up
in Fig.~\ref{fig:t2g-ge-Es}(b).



\com{
Summarise t2g results here.\\
copy from summary section...??
tilt vs rotation, preference:\\
 }


In a given material,
assuming
the charges on A and B are fixed
and either A or B dominates,
one type of distortion 
would be energetically preferred
over the other
for certain
numbers of electrons in 
the $d$-orbitals of the Jahn-Teller ion B.
For example,
if B cage dominates,
a $d^2$ configuration
should prefer tilting over rotation
because it stabilises two occupied orbitals 
while rotation stabilises one and destabilises the other occupied orbital.

 



\subsection{$e_{g}$ manifold}
\label{sec:eg-gen}


\com{
So we consider the general case of

both theta phi couples the two states..... expressions relatively longer, but, can be approximated by simpler expressions obtained by an expansion about zero angles.... 

simpler expressions of eigenstates and eigenvalues.... in terms of D2 and D4 terms of A and B.

presentation and discussion of eg splitting...
combined plots for the three cases w.r.t  angles... as three  rows
and two q values on the opposite side of the crossover...
so 2x3 panel figure.
}



Since $e_{g}$ has only two states,
we can always analytically diagonalise it
if we
ignore its coupling to the $t_{2g}$ states.
The expressions for the
matrix elements of the Hamiltonian are long however,
so we present the expansion at small angles in the following.
The diagonal terms with reference to the average $e_g$ energy are given by
\begin{align}
\label{eq:eg-hiix}
(H_{x^2-y^2},&H_{z^2}) = (1,-1)
\left[\frac{}{} \Omega_{2}  D_2 + \Omega_4 D_4\right],\\
\Omega_{l} =&  \left(q_A \Omega_{l}^A + q_B\Omega_{l}^B\right)/a,~(l=2,4),\\
\Omega_{2}^A =&  -\frac{64 \left(\alpha ^2 \left(7 \beta ^2+3\right)-6 \beta ^2\right)}{189 \sqrt{3}},\\
\Omega_{4}^A =& \frac{640 \left(\alpha ^2 \left(\beta ^2+16\right)-32 \beta ^2\right)}{1701 \sqrt{3}},\\
\Omega_{2}^B =&  \frac{3}{14} \left(\alpha ^2 \left(3 \beta ^2+2\right)-4 \beta ^2\right),\\
\Omega_{4}^B =& -\frac{5}{168}\left(\alpha ^2 \left(37 \beta ^2+46\right)-92 \beta ^2\right),
\end{align}
whereas the coupling between the two states is,
\begin{align}
\nonumber
H_{x^2-y^2,z^2}=
\frac{2\alpha ^2 \beta}{567} 
&\left(\frac{}{}128 q_A (3 D_2-20 D_4) \right. \\ \label{eq:eg-hijx}
&~~\left. - 81 \sqrt{3} q_B (2 D_2 -5 D_4)\right),
%&=\frac{2\alpha ^2 \beta}{567} \left(384 ( D_2-\frac{20}{3} D_4)q_A  - 162 \sqrt{3} (D_2 -\frac{5}{2} D_4)q_B \right)
\end{align}
which vanishes if 
either of $\alpha,\beta$ is zero.


%Similar to the $t_{2g}$ case, 
The average energy slightly shifts at nonzero $\alpha,\beta$
from $3/5\Delta$ by an amount
\begin{align}
\label{eq:eg-E0}
E_0^{e_{g}} &= \left(q_A E_0^A +q_B E_0^B\right)D_4/a - (3/5)f_{q_A}\Delta_o,\\
E_0^A &=
\frac{256 \left(5 \left(4 \beta ^2+1\right) \alpha ^2+5 \beta ^2-3\right)}{243 \sqrt{3}},\\
E_0^B &=
-\frac{1}{12} 5 \left(7 \beta ^2+4\right) \alpha ^2-\frac{5 \beta ^2}{3}+1.
\end{align}


% important, dont delete: H_{x^2-y^2,z^2} in terms of \Lambda_{l}^A/B
% where \Lambda_{l}^A/B are approximated upto second order in alpha beta.  
\com{
\begin{align}
H_{x^2-y^2,z^2}&=
\left[\frac{}{} \Lambda_{2}  D_2 + \Lambda_4 D_4\right],\\
\Lambda_{l} =&  \left(q_A \Lambda_{l}^A + q_B\Lambda_{l}^B\right)/a,~(l=2,4),\\
\Lambda_{2}^A =&
\frac{256 \alpha ^2 \beta }{189},\\
\Lambda_{4}^A =&
-\frac{1}{567} \left(5120 \alpha ^2 \beta \right),\\
\Lambda_{2}^B =&\frac{1}{7} (-4) \sqrt{3} \alpha ^2 \beta ,\\
\Lambda_{4}^B =&\frac{10}{7} \sqrt{3} \alpha ^2 \beta
\end{align}
}



\begin{figure}[htbp]
\begin{center}
%\includegraphics[width=1.0\linewidth]{/Users/maz/repos/perovskite/calc/analytical-structure-param/fig-data/energy-wts-eg}
\includegraphics[width=1.0\linewidth]{energy-wts-eg}
\caption{
Energies and characters of $e_g$ states.
$E_{\pm}$ and $E_0^{e_g}$
as a function of $\beta$ at $\alpha=10^\circ$, 
and the same parameters as in Fig.~\ref{fig:t2g-ge-Es}.
At $q_A=q_o/2$ (a,c) where B cage dominates,
$\alpha$ tends to make $\ket{E_{+}}$
more $x^2-y^2$ (see small $\beta$ region),
while $\beta$ tends to do the opposite 
(see large $\beta$ region).
The roles of $\alpha,\beta$ switch
at $q_A=2q_o$ (b,d), where A cage dominates.
We see that the splitting is relatively small as compared to $t_{2g}$ manifold.
Approximate results obtained by a second order expansion of the Hamiltonian given in Eqs.\ref{eq:eg-hiix}-\ref{eq:eg-E0}
are also shown for comparison.
}
\label{fig:e-wts-eg}
\end{center}
\end{figure}

Diagonalising the Hamiltonian,
we obtain two eigenstates $\ket{\pm}$
at energies $E_{\pm}$.
We present the results obtained using 
 the exact expressions instead 
of the expansions in Eqs.~\ref{eq:eg-hiix},\ref{eq:eg-hijx}.
Figure~\ref{fig:e-wts-eg}
shows $E_{\pm}$ and weights $W_{x^2-y^2}$ of $x^2-y^2$ orbital
in the two states
as a function of $\beta$
at $\alpha=10^\circ$, $\Delta_o=4q_o/15a$,
and 
$q_A/q_o=1/2,2$.
We get the same ordering of orbitals ``with and without $z$''
 as in $t_{2g}$ case,
i.e.,
at $q_A/q_o=1/2$,
$\beta$ tends to lower
$x^2-y^2$ while $\alpha$ tend to lower $z^2$ orbital,
as indicated by the behaviour of $W_{x^2-y^2}$,
which is almost zero in $\ket{E_{-}}$ at $\beta\lesssim 5^\circ$
 ($\ll \alpha=10^\circ$),
while it approaches unity
at $\beta\gtrsim \alpha$.
Similarly, at $q_A/q_o=2$,
$W_{x^2-y^2}$ shows this behaviour for $\ket{E_{+}}$.
This is also evident from the signs of the $\Omega_l$
in \eref{eq:eg-hiix}.
We see that $l=2$ terms have the same signs as $t_{2g}$ case but 
$l=4$ terms have switched the signs. 
Since $D_4$ is much smaller than $D_2$,
we will still get the same ordering as in $t_{2g}$ case.
The splitting is much smaller in this case though, only $\sim 0.5\%$ of $\Delta_o$
at $\beta=10^\circ$ in either case shown.
To see how well
the expansions in Eqs.~\ref{eq:eg-hiix},\ref{eq:eg-hijx}
can approximate the exact results,
Fig.~\ref{fig:e-wts-eg} includes
them as thin dotted lines that are collectively labelled Approx.,
 which are almost indistinguishable from the exact results
at $q_A/q_o=1/2$, but deviate at $q_A/q_o=2$ as their
``resonance point'' is shifted towards larger $\beta$.



\subsection{Effect of the full lattice}
\label{sec:FL}

In a crystal, obviously, not only the nearest coordination cages of A,B ions but also
 the other ions (including O) at longer distances
all have finite contributions to the crystal field and hence the JT effect.
To see how much our results change when 
%the potential of 
 the full lattice (FL) is considered instead of just the nearest AB cages,
we numerically calculate the multipole components of the potential
 of the full lattice by following the recipe in Ref.~\cite{FinnisPRL98,PaxtonPRB08,PaxtonNotes}
and Ewald summation method~\cite{Ewald1921}.
To do this, 
we borrow relevant routines from the tight binding package in Questaal suite~\cite{Questaal} that implements these methods.


\com{First,
FL does not preserve the cubic symmetry of the potential and the degeneracy of the t2g manifold 
is slightly lifted at zero rotations, i.e., for a cubic structure.
However, it is insignificant and we will not discuss it further.
What's more important is that the
potential of the full lattice does not change the qualitative picture we resented
but merely shifts the balance between A and B cages towards smaller $q_A$.
}

We find that the
potential of the FL does not change the qualitative picture we presented
but merely shifts the balance between A and B cages towards smaller $q_A$.
This can be seen in Fig.~\ref{fig:t2g-lattice}
where the energies of the $t_{2g}$ states as a function of $q_A/q_o$
are compared for the two cases
at $\alpha=0$, $\beta=20^\circ$ (and $\Delta_o=4q_o/15a$).
In case of AB cages, the crossing of the 
($xy$ and $yz/zx$)
energy levels occurs 
at $q_A/q_o\simeq1.1$,
whereas the FL moves it to $q_A/q_o\simeq0.3$.
The slopes of the lines are almost the same in both cases,
meaning that the splitting due to octahedral rotations 
would be the same in both cases if $q_A$ is measured 
from their crossing points.
The shift towards smaller $q_A$ in case of FL
can be explained by noting that
the next coordination cage beyond the first AB cages
has O ions
($24$ of them at a distance of $\sqrt{5}a/2$ at $\alpha,\beta=0$) %and $\sqrt{5/2}a$ at $\alpha,\beta=0$),....
%\az{both of which supports the A cage potential.??? plot and see... }
that supports A cage
against the B cage, lowering the value of $q_A$ that balances it
at the crossing in Fig.~\ref{fig:t2g-lattice}.
It significantly increases the agreement between the two cases presented
by moving the crossing point 
to $q_A/q_o\simeq0.8$ (not shown in Fig.~\ref{fig:t2g-lattice}).

From the above results, we also see that
a comparison between the rotation induced splittings
 in the two
cases (FL vs AB cages)
at a fixed value of $q_A$
would show an underestimation or an overestimation 
depending on whether $q_A$ lies to the right or the left side of the 
middle point of the two crossings.
Besides, the characters of the eigenstates
would be in a disagreement at $q_A$
between the two crossings.




\com{
Since the qualitative picture stays the same
in FL case,
we can 
think of the effect of the rest of the ions in the lattice
still in terms of the AB cages by renormalising $q_A$.
}




\begin{figure}[htbp]
\begin{center}
%\includegraphics[width=1.0\linewidth]{/Users/maz/repos/perovskite/calc/analytical-structure-param/full-lattice-potential-tests/full-lattice-vs-AB}
\includegraphics[width=1.0\linewidth]{full-lattice-vs-AB}
\caption{
The effect of the potential of full ionic lattice.
$t_{2g}$ spectrum computed using the potential of full lattice 
and that obtained from only nearest AB cages (and the anion octahedron itself)
are
shown as a function of $q_A/q_o$ at 
$\alpha=0,\beta=20^\circ$, $\Delta_o=4q_o/15a$.
Full lattice only shifts the balance between the two cation cages towards A 
(i.e, the crossing point, where the two cages balance each other, moves to smaller $q_A$)
but keeps the qualitative picture the same.
(The bold black line is nothing but an overlap of two practically degenerate black lines.)
}
\label{fig:t2g-lattice}
\end{center}
\end{figure}



\section{Summary and Conclusion}


We consider the effects of octahedral rotations and tilting in perovskites
on the $t_{2g}$ and $e_g$ manifolds of the central transition metal ions.
We find significant splitting in these levels
caused by the distortions in the crystal field of nearest A and B ion cages,
which
essentially amounts to a 
Jahn-Teller instability.
The sign and strength of this effect is 
determined by
a competition between the two ion cages.
%depending on the proportion of charges $q_A,q_B$ on A and B ions.
The tilt 
%$\alpha$ 
and rotation 
%$\beta$
also tend to establish different ground states
%that depends 
depending
on which ion cage dominates.
While the tilting alone can mix $t_{2g}$ states,
$e_g$ states
do not mix unless both rotation and tilting are non-zero.
Despite such mixing, we can see which orbitals are stabilized or destabilized.
If A cage dominates (i.e., $q_A$ is larger than a critical value),
%$\alpha$
tilting lowers
$xy,x^2-y^2$ orbitals
and raises $yz,zx,z^2$,
while 
%$\beta$
rotation tends to do the opposite;
the whole situation reverses if B cage dominates.
Considering the crystal field of the full lattice 
only shifts the balance towards A cage
but does not qualitatively change the results.



\section{Discussion and Outlook}
\label{sec:discussion}


\com{
% beside the hardness, the Madelung energy will increase due to an increased repulsion between the two ions that take part in the charge transfer. Consider that q1^2+q2^2 with constant that q1+q2=2q=fixed is min when q1=q2=q.
% so the electronic energy gain will have to fight against both the chem hardness and the Madelung energy.
Based on our analysis of A and B cages,
we expect that,
in case of atomic scale perovskite superlattices
with cations having different charges in alternating layers,
the crystal field at B sites will bear the exact same deformation as in case of 
a tetragonal distortion of the octahedra,
and should lead to 
a JT like splitting even 
in the symmetric structure, i.e., with no structural distortion.
Considering the charge as a dynamic variable,
it should then also be possible that in 
simple perovskites structure,
the system might encounter charge instabilities similar to the structural ones,
where the system is stabilised by a charge transfer from one site to another
that lowers the electronic energy and costs 
the ``elastic'' energy proportional to the chemical hardness of the ions involved.
Such considerations might be useful in studying polaron formation and dynamics.
}



% orbital ordering
We have only focused on the spectrum of 
the B ion at 
%position ${(0,0,0)}$
$\textbf{a}/2$
 in the cell 
described in Sec.~\ref{sec:structure}
but 
there are three more \emph{distinct} B ions in the unit cell 
at finite octahedral rotations that also 
need to be considered simultaneously.
We find that the splittings in the spectra
of all B ions are
practically identical.
However, their eigenstates 
are different in a way that can
create interesting orbital ordering
that depends on the tilt and rotation angles,
unlike what happens in 
tetragonal Jahn-Teller distortions (elongation and compression) 
such as seen in KCuF$_3$~\cite{LiechtensteinPRB95}, 
where the orbital ordering does not depend on the size of the distortion.
The eigenstates' components in the local bases attached to the octahedra's frame
also vary between different B sites 
so the dependence on the rotation and tilt angle is not trivial,
which is understandable because different B sites in the unit cell have different environments. 
A detailed study of orbital ordering in this case is out of scope of this manuscript and is left for future work.


Of course, beside the crystal field we consider in our simplistic model,
in real materials,
 there would be other local effects like
electronic correlations (Hubbard and Hund's interaction)
and spin-orbit interaction, 
as well as band structure effects arising from
the electron hopping between the B ions.
The interplay between these and the crystal field (or JT) effect 
we discussed in this manuscript is also an interesting future work.
A discussion 
about the role of 
electron hopping 
along with some preliminary results 
follows.



Assuming the crystal field splitting to be the largest energy scale,
the orbitals in the $e_g$ and $t_{2g}$ manifolds of B sites are aligned with their respective octahedra.
The electron hopping between these orbitals that are 
localized on different B sites 
%(that creates electronic bands)
depends on the relative orientation of the orbitals.
As the octahedra rotate (or tilt),
the hybridization between the
neighboring sites
change,
modifying
the electronic bands and associated band energy.





\begin{figure}[htbp]
\begin{center}
\includegraphics[width=1.0\linewidth]{hybrid}
\caption{The effect of electron hopping between B sites.
(a) Energy of two configurations $t_{2g}^1$ and $e_g^1$
as a function of $\beta$ when either the crystal field or the hopping is considered alone.
In the latter case, the energy changes are scaled with the bandwidth $W$.
(b) Considering both the crystal field and the hopping simultaneously,
the ratio of the change in energy with rotation $\beta$ at a finite hopping to that in the absence of hopping $E(W)/E(0)$
as a function of the average bandwidth $W$, 
for the same two configurations
$t_{2g}^1$ and $e_g^1$.
The results of the crystal field model presented in this manuscript 
hold in the small bandwidth limit where $E(W)/E(0)\sim 1$.
}
\label{fig:hybrid}
\end{center}
\end{figure}



To shed some light on the 
role of these band structure effects,
we consider the nearest neighbor hopping between the B sites
using the Slater-Koster method~\cite{SlaterPR54}
and
calculate the change in the energy of 
the system
as the octahedra rotate.
We do this for three cases: with either the crystal field or the hopping only,
 and with both of these simultaneously.
%
Figure~\ref{fig:hybrid}(a)
shows the change in the energy 
of two (spinless) configurations $t_{2g}^1$ and $e_g^1$
as a function of $\beta$ for the first two cases
for the crystal-field-only (black curves, left y-axis;
$\alpha=0$, $q=2q_o$, $\Delta_o=4q_o/15a$)
and for the hopping-only case (red/gray curves, right y-axis).
We
use Slater-Koster hopping parameters
$(t_{\sigma},t_{\pi},t_{\delta})=(0.15,-0.1,0.025)$~eV, 
% W = \{0.0330745,0.0385869,0.0358307\} Hartree
% W = {0.900005, 1.05, 0.975004} eV
%$(t_{\sigma},t_{\pi},t_{\delta})/W= {0.153846, -0.102564, 0.0256409}$
$a=7$~bohr, $q_o=2$~e,
and consider only the crystal field of O octahedra that creates
$t_{2g}$ and $e_g$ manifolds.
(These parameters produce an average
bandwidth $W=0.97$~eV for the two manifolds).
While both $t_{2g}^1$ and $e_g^1$ configurations 
show a lowering of energy 
due to
the crystal field 
as $\beta$ 
increases---consistent with the results presented 
before---the energy of $e_g^1$ increases when only hopping is considered 
(the effect scales with the hopping parameters or the bandwidth).
This is due to a reduced hybridization leading to narrower $e_g$ bands.
In contrast, the energy of $t_{2g}^1$ decreases due to a band widening arising from 
an enhanced hybridization.
The crystal field and these band structure effects are not additive however.
The rotation induced splitting of the $t_{2g}$/$e_g$ manifolds
arising from the crystal field
suppresses the effectiveness of the hopping in creating 
the dispersion in the energy states, 
and thus reduces the bandwidths and the
the associated rotation induced energetic changes.


To show the validity regime of our crystal field model 
presented in this manuscript,
Fig.~\ref{fig:hybrid}(b)
presents  
the ratio of the
 change in the energy 
at $\beta=20^\circ$ (i.e., the energy difference from the cubic case)
at finite hopping $E(W)$ to that of our simplistic crystal field model $E(0)$
as a function of the average bandwidth $W$ of $t_{2g}$ and $e_g$ bands.
Here we scale the Slater-Koster parameters used for Fig.~\ref{fig:hybrid}(a)
to tune the bandwidth $W$.
We see that $E(W)/E(0)$ decreases from unity at $W=0$
to half its value for $t_{2g}^1$ configuration at $W/\Delta_o\gtrsim0.3$,
and even becomes negative for $e_g^1$ configuration around 
$W/\Delta_o\gtrsim0.15$ (showing the dominance of the band structure effects).
From this analysis, we conclude that the results of our crystal field model 
hold unambiguously only for narrower bands, $W/\Delta_o\ll1$,
while a detailed further study is required to understand the interplay of the two effects
at larger bandwidths.











%\begin{acknowledgements}
%\end{acknowledgements}

\appendix*
\section{Computing $V_{lm}$}
\label{append:multipoles}

Here we describe a method to obtain the multipole components 
$V_{lm}$ of a potentials $V(\vec r)$ %of $A/B$ ion cages 
at a position $\vec r$ around a JT ion.
It is worth noting that this can be done without 
evaluating the 
inner products between the spherical harmonics and
$V(\vec r)$.


Let's consider $V^{A}(\vec r)$ to illustrate the idea. 
$V^{B}(\vec r)$ and $V^{JT}(\vec r)$ can also be dealt in the same way.
We like to expand the potentials $V^{A}(\vec r)$
in spherical harmonics as
\begin{align}
\label{eq:ylm_compA}
V^{A}(\vec r) =& \frac{q_A}{a}\sum_{l,m} V_{lm}^{A} \left(\frac{r}{a}\right)^l Y_{lm}(\hat r),
\end{align}
where we have assumed $a$ as a length scale and 
taken out the factor $q_A/a$ for clarity.

First we calculate
$V^{A}(\vec r)$
using the simple textbook formula.
$V^{A}(\vec r;\alpha,\beta)=q_A\sum_{i}1/|\vec r^A_i-\vec r |$, 
where $\vec r^A_i$ is the position of $i$-th A ion
and 
$\vec r$ is the position of the observation point 
measured from the JT ion.
Writing cartesian components of $\vec r$ as
$\vec r=r(\cos\phi\sin\theta,\sin\theta\sin\phi,\cos\theta)$,
where $(r,\theta,\phi)$ 
are the spherical coordinates in the frame of JT ion,
and expanding 
${V^{A}(\vec r;\alpha,\beta)}$
in powers of $r$, 
we can separate 
coefficient of $r^l$
that give degree $l$ components.
We can then consider each degree $l$ term individually,
and resolve it in its $(2l+1)$ $m$-components.
This can be done by 
expanding such a coefficient of $r^l$
in powers of $\cos\theta, \cos\phi$ 
(or $\sin\theta, \sin\phi$) 
and comparing the terms with a similar expansion of 
the formal expression on the right side 
of \eref{eq:ylm_compA}.
This gives a set of
linear equations in $\{V_{lm}^{A}\}$ (for a fixed value of $l$)
 that can be easily solved.
The whole calculation can be performed
 on Mathematica~\cite{Mathematica}
or a similar software.



\label{Bibliography}
%\bibliographystyle{apsrev4-1}
\bibliography{perovskites}


\end{document}
