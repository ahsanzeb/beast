
\documentclass[a4paper,prb]{revtex4-1}  %a4paper,prb,twocolumn
%\documentclass[preprint]{revtex4-1}

%\documentclass[12pt]{report}

%\usepackage{orcidlink}

\usepackage{graphicx}  % needed for figures
\usepackage{dcolumn}   % needed for some tables
\usepackage{bm}        % for math
\usepackage{amssymb}   % for math
\usepackage{hyperref}

\usepackage{soul,color}

\usepackage{caption} %For two images side by side
\usepackage{subcaption} %For two images side by side

\usepackage{amsfonts}

\usepackage{amsmath}
%\usepackage{kbordermatrix}
\usepackage{blkarray}
\usepackage{braket}
\usepackage{multirow}
\usepackage{mathtools}

\usepackage{enumerate}

% for figure (Cartoon)
\usepackage{tikz}
\usepackage[utf8]{inputenc}

\newcolumntype{L}{>{$}l<{$}} % math-mode version of "l" column type


\newcommand{\y}{\lambda}
\newcommand{\yp}{\lambda^\prime}
\newcommand{\ph}{\Phi}
\newcommand{\ps}{\Psi}
\newcommand{\w}{E}
\newcommand{\cw}{\Omega}
\newcommand{\dl}{\delta}
\newcommand{\x}{\times}
\newcommand{\llra}{\Longleftrightarrow}
\newcommand{\ep}{\varepsilon}
\newcommand{\sig}{\sigma}
\newcommand{\al}{\alpha}
\newcommand{\be}{\beta}

\newcommand{\eps}{\epsilon}
\newcommand{\wrr}{\Omega_R}

\newcommand{\sx}{\mathcal{\hat S}_0}
\newcommand{\sy}{\mathcal{\hat S}_k}


\newcommand{\N}{\mathcal{N}}
\newcommand{\nb}{N}
\newcommand{\nd}{N_D}
\newcommand{\Nd}{\mathcal{N}_D}
\newcommand{\Nb}{\mathcal{N}_B}



\newcommand{\stt}{\hat{\mathcal{S}}}
\newcommand{\stf}{\hat{\tilde{\mathcal{S}}}}

\newcommand{\ws}{{\omega^\prime}}

\newcommand{\ev}{\hat{\Psi}^+(\w)}



\newcommand{\stket}[1]{{\mathcal{S}_{#1}}}

\newcommand{\nex}{\hat{\mathcal{N}}_{ex}}

\newcommand{\h}{\mathcal{\hat H}}
\newcommand{\z}{\mathcal{Z}_{\alpha}}
\newcommand{\hi}{\hat h_{\alpha}}
\newcommand{\hj}{\mathcal{\hat H}_j}

\newcommand{\hd}{\mathcal{\hat H}_{\Delta}}

\newcommand{\com}[1]{}

%\DeclareMathOperator{\Tr}{Tr}
\DeclareMathOperator{\diag}{diag} 
\DeclareMathOperator{\integer}{integer}
%\DeclareMathOperator{\nd}{and}
\DeclareMathOperator{\eV}{eV}
%\everymath{\displaystyle}

\newcommand{\ad}{\hat{a}^\dagger}
\newcommand{\an}{\hat{a}^{}}
\newcommand{\sd}{\hat{\sigma}^{+}_i}
\newcommand{\sn}{\hat{\sigma}^{-}_i}

\newcommand{\ua}{\uparrow}

\newcommand{\fd}{f(\hat{b}^\dagger)}
\newcommand{\fn}{f(\hat{b}^{})}


\newcommand{\mbf}[1]{\mathbf{#1}}
\newcommand{\del}{\hat{\Delta}^{}}



\newcommand{\fref}[1]{Fig.~(\ref{#1})}
\newcommand{\tref}[1]{Table~\ref{#1}}
\newcommand{\eref}[1]{Eq.~(\ref{#1})}

\newcommand{\rev}[1]{{\color{blue}{#1}}}  
\newcommand{\az}[1]{{\color{magenta}{#1}}} %Ahsan Zeb
\newcommand{\ha}[1]{{\color{red}{#1}}} 
\newcommand{\sm}[1]{{\color{brown}{#1}}} 

\newcommand{\Tau}{\mathrm{T}}

\DeclareMathOperator\erf{erf}
\DeclareMathOperator\erfi{erfi}
\DeclareMathOperator\ArcTanh{ArcTanh}
\DeclareMathOperator\sgn{sign}

\newcommand{\ylm}[1]{Y_{#1}(\hat r)} 


\def\Xint#1{\mathchoice
   {\XXint\displaystyle\textstyle{#1}}%
   {\XXint\textstyle\scriptstyle{#1}}%
   {\XXint\scriptstyle\scriptscriptstyle{#1}}%
   {\XXint\scriptscriptstyle\scriptscriptstyle{#1}}%
   \!\int}
\def\XXint#1#2#3{{\setbox0=\hbox{$#1{#2#3}{\int}$}
     \vcenter{\hbox{$#2#3$}}\kern-.5\wd0}}
\def\ddashint{\Xint=}
\def\dashint{\Xint -} % \sim


\newcommand\scalemath[2]{\scalebox{#1}{\mbox{\ensuremath{\displaystyle #2}}}}

%\usepackage[paperheight=18cm,paperwidth=14cm,textwidth=12cm]{geometry}
%\usepackage{setspace}
%\doublespacing

\begin{document}

\begin{center}
\textbf{BK14571/Naheed: Jahn-Teller effect with rigid octahedral rotations in perovskites}
\end{center}


Dear Editor,
\\ 

We thank you and the referee for their invaluable insight and comments.
\\

We have revised our manuscript in response to the referee's comments. 
We feel that the revised version has been improved significantly, especially due to a discussion of the hopping effects and associated results in newly added Fig.~8.
\\

A point by point response 
to the the referee's comments
along with a description of the changes made is given below at the end.
\\

We hope that the referee would be satisfied with the revised version and the manuscript would kindly be granted acceptance. 
\\

Thank you!
\\

Yours Sincerely,
\\

The Authors
\\

\rule[1ex]{\textwidth}{.5pt}

\textbf{Major changes:}
\begin{itemize}
\item
Included a discussion on the hybridization effects at the end of Sec.~VI along with a new figure, Fig.~8. (page 11 in the revised version)
\end{itemize}


\textbf{Minor changes:}
\begin{itemize}
\item 
Added a comment ``(remember that we take anion charge to be $-q_o$)''
just above Eq.~(16) on page 4.
\item
Rewritten the second half of the first paragraph 
of Sec.~IV on page 6. Typos corrected.
\item
Added to the caption of Fig.~7:
 ``(The bold black line is nothing but an overlap of two practically degenerate black lines.)''.
 \item
 Added labels of A,B cages in Fig.~1(c,d).
 
 \item 
References style changed: titles of the journal articles are included.
\end{itemize}
\rule[1ex]{\textwidth}{.5pt}
\\







\textbf{Referee's comments (blue) and our response (black):}
\\

\rev{
The paper "Jahn-Teller effect with rigid octahedral rotations in
perovskites" by R. Naheed and co-authors represents in my opinion a
very nice piece of work, where the authors thoroughly study a rather
complicated subject of influence of rotation and tilting on lifting
orbital degeneracy. However, it has to be mentioned that there is one
rather general point, which must be cleared out before the decision
can be made.

The present treatment follows a general route often taken in the JT
physics: to consider the effect of Madelung field, but there is also
another extremely important contribution to the crystal field coming
from the hybridization, i.e. the overlap between orbitals centered on
different sites. Typically it acts in the same direction, but this
effect should be taken into account for the quantitative results, on
which the authors aimed. The angle dependence of the hopping
integrals, can be considered via the Slater-Koster approximation.
Other, more technical comments:
}

Thanks for your kind and positive opinion,
and mentioning your concern about the hybridization effects.
We have done some preliminary calculations 
including the hopping between the B sites
and find that they can significantly affect our results when
the bandwidth of the $t_{2g}$ and $e_g$ bands cannot be ignored in comparison to the crystal field splitting between the two manifolds. 
This limits the validity of our crystal field model to small bandwidths.

We find that the crystal field and the hopping do not produce additive effects.
We have included the main idea
in the discussion section in the revised version
(Fig.~8 and last three paragraphs of sec.~VI). A brief description of it follows.
When considered separately, 
the crystal field (Madelung) effects are scaled with the crystal field splitting
whereas the hybridization effects are scaled
with the bandwidth [Fig.~8(a)].
However,
when considered simultaneously,
lifting of the degeneracy due to the crystal field 
suppresses the energetic changes due to the hybridization effects [Fig.~8(b)].
A detailed study of the interplay between the two types of effects is left for future work.


\begin{enumerate}
\item
\rev{It is not clearly explained how exactly the authors define sign of
q0, qA,qB. E.g. the charge neutrality relation mentioned above (16) is
$qB = 3q0 - qA$. For oxides q0 is -2 and $qA>0$ and does not typically
exceed +4, but then $qB<0$. Obviously the authors used some other
conventions, but do not really explain this.
}

Thanks for pointing it out.
We take the charge on the anion/oxygen to be $-q_o$ (so $q_B=3q_o-q_A$ is not incorrect).
This is stated at the start of  
Sec.~I-A: Atom/ion in an octahedral field: $t_{2g}$  and $e_g$ manifolds (page 3, first column).
For oxides, $-q_o=-2$ (so $q_o=2$), and the charge netrality correctly gives $q_B=6-q_A$;
$q_A, q_B >0$. 


In the revised version,
for clarity,
we have added a comment ``(remember that we take anion charge to be $-q_o$)''
just after the charge neutrality relation above Eq.~(16).


\item
\rev{
I'm not sure that it is clear for a general reader why from volume
dependence $V = a^3 (\delta_1 + \delta_2)$ follows that the elastic energy
must be proportional to $(\delta_1 + \delta_2)^2$ [BTW one of the
subscripts is wrong]. OK, they know from the high-school courses that
the elastic term must quadratic and could agree, but why the same
logic applies to $(\alpha^2 + \beta^2)^2$ can be not clear. Moreover,
it's not clear a priori why there is no cubic terms. Please explain
this part in more detail.
}

Thanks. 
The the main point we
tried to make here is that 
the electronic and elastic energies have different 
polynomial dependence on the distortion (so the optimum structure is guaranteed to be distorted).
For our purpose, it is not important to know the exact form of the elastic energy, and whether the anharmonic effects would be important or not.

To avoid any distraction and confusion,
we have rewritten the paragraph. 
We now focus
on conveying the key point without
unnecessary details.




\item
\rev{
Fig. 7. Should not be there explained what a bold line means? And
why there is no a bold line for dashed curves?
}


Thanks for noticing. 
What appears to be a bold line are actually two nearly degenerate lines of the same thickness as the other lines in the plot.
In comparison to the dashed lines (nearest AB cages only), 
the Ewald sum
in case of infinite lattice considers
a large number of \emph{unit cells}
that include some \emph{incomplete}
coordination cages at the boundary, 
 which breaks some symmetries present in the 
 case of complete coordination cages.
 Since the two lines are practically degenerate,
 we did not check if there are other reasons apart from the above symmetry breaking (e.g., some noise from O ions that belong to other octahedra nearby).


We think that, in the manuscript,
it suffices to simply clearly state that these are two lines
and leave the 
above technical details. 
So, we have included in the caption:
 ``(The bold black line is nothing but an overlap of two practically degenerate black lines.)''.

\item
\rev{
I would also suggest to put labels of the A and B cages to Fig. 1.
}

Thanks. We have done it in the revised version.

\end{enumerate}





\end{document}